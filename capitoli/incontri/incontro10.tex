\subsection*{Sommario}

In questo incontro S. introduce un nuovo tema, il proprio rapporto con il \index{marito!apporti con}marito. A differenza dei primi incontri, il racconto è puntuale e focalizzato, e quando si tratta di indicare le problematicità, S. è puntuale e precisa, anziché debordante. In particolare, non fa più menzione di quella ''ansia''\index{ansia} che all'inizio del caso sembrava essere la sua prima e principale preoccupazione.

\subsection*{Frasi salienti}

Inizialmente, S. riporta alcune sue riflessioni riguardo al cambiamento\index{cambiamento} che vede in sé da quando sono iniziati questi incontri.

\begin{verse}
S: più ci penso più mi accorgo di essere stata una gran sfruttatrice\\
C: cosa intendi con ''sfruttatrice''?\\
S: che ho sempre preferito dare agli altri la \index{responsabilità}responsabilità anche delle cose che toccava a me decidere\ldots{} cioè non ''sempre'' in senso letterale, perché per esempio quando ho voluto andare al Liceo mi sono ben impuntata. Però quando c'era in ballo la mia paura di sbagliare, o di essere giudicata, allora sì, mi nascondevo, lasciavo che gli altri decidessero per me. Molto comodo, non rischi niente\ldots ma il risultato è che ho vissuto una vita che non era mia, era quella che pensavano gli altri.\\
C: chi sono questi ''altri''?\\
S: mio padre, qualche volta perfino i miei fratelli, poi mio \index{marito}marito, poi i superiori sul lavoro,\ldots{} la lista è lunga\\
C: vuoi parlarmene un po' più in dettaglio?\\
S: ma no guarda, non mi interessa andare a scavare il passato, davvero. Cosa è successo l'ho capito. L'ho capito parlando dell'ansia\index{ansia} che avevo con mio \index{marito}marito, con mio figlio piccolo. Il meccanismo l'ho capito, non mi interessa andare a vedere quante volte avrei potuto agire in modo diverso\ldots{} mi interessa di più stare attenta a non ricadere nello stesso meccanismo. Ecco, adesso sono molto più attenta a tante cose, e vedo che ci sono altre cose che non mi va più il modo in cui vanno.\\
C: a cosa ti riferisci in particolare?\\
S: il rapporto con mio \index{marito!rapporti con}marito\\
\end{verse}

\noindent S. introduce il nuovo tema dei rapporti con il \index{marito!rapporti con}marito, e dell'evoluzione che hanno avuto negli ultimi mesi, anche grazie al fatto che S. ha imparato a gestire la propria ansia\index{ansia}.

\begin{verse}
C: cosa mi dici del rapporto con tuo \index{marito!rapporti con}marito?\\
S: che da un lato, guarda, in questi ultimi mesi è migliorato. Non solo perché ho una serata per me, anche se poi non me la prendo sempre, ma proprio a livello di rapporto. Prima litigavamo spesso, ogni volta che lui partiva con quel modo di fare condiscendente, quel ''fai, fai, tanto\ldots '' io scattavo come una molla e a quel punto era tutto un rinfacciare. Invece adesso le cose sono più tranquille, da quel punto di vista\\
C: cosa intendi con ''più tranquille''?\\
S: ultimamente siamo più come amici, riusciamo a parlare anche dei problemi, la casa, i \index{figli}figli, senza darci le colpe. In un certo senso è come se io fossi meno dipendente da lui e lui meno dipendente da me. Questi anni di counseling mi hanno fatto molto bene, sono più sicura, riesco a esprimermi molto meglio\ldots e anche in questi mesi gli ho parlato spesso delle cose che scoprivo negli incontri, di cosa ne pensavo, delle cose nuove che vedevo. Lui all'inizio pensava che fosse una delle mie solite idee balzane, ma adesso vede che in certe cose proprio sono cambiata.
\end{verse}

\noindent S. descrive come è cambiato il proprio rapporto con l'ansia\index{ansia} che ha rappresentato il tema dominante dei primi incontri. Credo che questo rappresenti un punto di arrivo, una sorta di conclusione naturale di questo ciclo di incontri. Chiedo a S. cosa ne pensa lei.

\begin{verse}
S: tutta quell'ansia che avevo sempre, no?, adesso non è che è sparita, però mi accorgo che c'è e riesco a \emph{scioglierla}, e questo mi rende molto più serena.\\
C: se dici così mi sento contento, e penso che abbiamo raggiunto un \index{obiettivo!raggiungimento del}obiettivo di cambiamento\index{cambiamento} a lungo termine, oltre ai risultati pratici degli obiettivi che ci eravamo dati. Tu cosa ne pensi?\\
S: io penso che già non avrei creduto di poter chiedere una serata, meno che mai ottenerla, e tantomeno abbandonare l'idea di dover proteggere [nome del figlio piccolo] dalle sue \index{responsabilità}responsabilità, dalla sua vita. Ma la cosa che mi colpisce di più, sì questi sono stati risultati importanti, ma la cosa che mi colpisce di più sono le ricadute che ci sono state su tutto il resto. Potere finalmente trovare una via d'uscita da quell'ansia che avevo, soffocante\ldots{} ora riesco a vedere le cose sotto un'altra luce, a viverle senza avere sempre l'angoscia. Non dico di essere diventata un'altra, ma c'è stato un \index{cambiamento}cambiamento, e se ne accorgono anche gli altri.\\
\end{verse}

\noindent Mi riallaccio al tema dei rapporti con il \index{marito!rapporti con}marito, per capire come sono cambiati a fronte del \index{cambiamento}cambiamento di cui S. ha appena parlato.

\begin{verse}
C: anche tuo \index{marito}marito nota questo cambiamento?\\
S: eccome! Se pensi che prima l'ansia\index{ansia} mi prendeva per qualsiasi cosa, capisci che anche lui una differenza la vede eccome, e pure grossa.\\
C: e tu, che differenza vedi nel modo in cui vi rapportate?\\
S: vedi, io prima ero tutta concentrata sul fatto che lui non mi capiva, non mi aiutava, gli rinfacciavo tutta una serie di cose, anche coi ragazzi\ldots{} Adesso no, ti dicevo che riusciamo a parlarci molto meglio, cerchiamo assieme le soluzioni\ldots{} è meglio ma in un certo senso il problema è più serio [pausa] perché io adesso mi accorgo che il problema non erano i litigi, il rinfacciarsi, perché sotto c'era dell'altro [lunga pausa] io adesso non so più se sono innamorata di lui, e lui di me\\
C: cosa intendi con ''non so più se sono innamorata''?\\
S: che un conto è non litigare, la pace in famiglia, adesso siamo una famiglia meno litigiosa, certo, ma come coppia\ldots{} come coppia non so se siamo più una coppia; lui non mi stimola, vedo che abbiamo visioni della vita molto diverse, io vorrei viaggiare, finché ci sono ancora i ragazzi in casa, portarli a vedere un po' di mondo, lui invece è più tranquillo, gli dai il suo calcetto il venerdì, poi ha il suo lavoro\ldots sembra che non senta bisogno di altro. E anche a livello fisico [pausa] non c'è più desiderio. Lui sembra non farci caso, mentre io ho bisogno del contatto, io\ldots non credo che ci sia più qualcosa fra noi, a parte la famiglia\\
C: se capisco bene vorresti trovare un \index{obiettivo}obiettivo di cambiamento per questa situazione?\\
S: io vorrei trovare il coraggio di dirgli ''è finita''\ldots{} ma non sono sicura di riuscire a farcela, soprattutto non adesso\\
C: in che senso ''soprattutto non adesso''?\\
S: [lunga pausa] a lungo termine forse sì, ma adesso, i ragazzi sono in un'età difficile, il grande è appena adolescente, l'altro lo sarà fra poco\ldots non è il momento di dargli uno scossone del genere. E poi, tutto sommato, ora che non litighiamo quasi più, c'è molta più serenità. Non dico che si debba restare assieme solo per i \index{figli}figli, ma se si può fare in modo di fare tutto nei tempi giusti, in modo che per loro non sia traumatico\ldots{} ti dico, adesso andiamo molto meglio di due, tre anni fa. Molto meglio, anche i ragazzi se ne sono accorti. Siamo una famiglia serena; collaboriamo per dare ai ragazzi un'educazione, una direzione, non ci rinfacciamo più le cose. Io credo che questo sia già molto. A me, adesso, la situazione con lui non pesa, non in questo momento. Magari  più avanti, ma adesso mi sto godendo questa nuova serenità, mi va bene così, voglio prendermi il tempo per abituarmi all'idea, poi magari facciamo un'altra serie di incontri.
\end{verse}

\noindent Faccio una \index{tematizzazione}tematizzazione riassuntiva: i risultati raggiunti, la nuova capacità di sciogliere l'ansia, le conseguenze positive sui rapporti con la famiglia. Riassumo anche quanto S. ha detto riguardo al suo rapporto con il \index{marito!rapporti con}marito: il superamento dei litigi, la scoperta di poter collaborare alla conduzione della famiglia, e la scoperta che i litigi nascondevano un problema più profondo: la probabile fine del rapporto di coppia fra S. e il marito. Concludo la tematizzazione dicendo che i risultati raggiunti costituiscono una buona conclusione naturale di questo ciclo di incontri e che S. vuole prendersi del tempo per abituarsi alle nuove problematiche che ha scoperto, che potranno costituire l'argomento di una prossima serie di incontri.
