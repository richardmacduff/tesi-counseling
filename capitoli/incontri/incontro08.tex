\subsection*{Sommario}
Questo incontro ha luogo quasi un mese dopo il precedente. In questo periodo, ho avuto con S. un paio di contatti telefonici riguardo nei quali ci siamo aggiornati sull'evolversi della situazione con i compiti di Italiano del \index{figlio!secondo}figlio minore, e con la rispettiva insegnante. S. ha espresso ripetutamente la sua titubanza a ridurre il proprio intervento nello svolgimento dei compiti ma, allo stesso tempo, di avere maturato la convinzione che il proprio comportamento al riguardo non possa continuare come prima. Si ripromette di affontare il problema non appena avremo un altro incontro.

In questo incontro, perciò, definiamo con S. un ulteriore obiettivo di \index{cambiamento}cambiamento, stavolta relativamente al rapporto con il \index{figlio!secondo}figlio minore, poi chiedo a S. come sono cambiate le cose rispetto ai primi incontri, nei quali l'ansia\index{ansia} era il tema costante.
L'incontro apre una finestra su un altro tema sul quale sta riflettendo in questo periodo: il suo rapporto con il \index{marito!rapporti con}marito. Il dialogo è però in forma interlocutoria, senza che S. indichi un problema specifico che desidera affrontare.

S. appare in grado, rispetto ai primi incontri di analizzare le proprie sensazioni con maggiore distacco, senza subirle passivamente. 

\subsection*{Frasi salienti}

L'incontro inizia riprendendo il tema dell'incontro precedente ed elaborando un'obiettivo di azione da compiere a breve termine.

\begin{verse}
S: \ldots ci ho riflettuto molto, con [nome del figlio minore] non riesco a sbloccare la situazione, fa sempre arrabbiare per lo studio, non si applica\ldots\\
C: quando lui ''fa'' arrabbiare, chi è che si arrabbia?\\
S: beh, io, è ovvio [sorride]\\
C: e la sua reazione, invece quale è?\\
S: ah, lui sbuffa, si distrae, non si applica, \ldots\\
C: quando ci sono i compiti di Italiano, ti chiama lui per aiutarlo?\\
S: no, controllo il diario e glielo dico io\\
C: quindi mi stai dicendo che lui non chiede il tuo aiuto?\\
S: no, ma io lo seguo lo stesso perché, come ti dicevo la volta scorsa, se poi mi rimane indietro\ldots [lunga pausa] è dove siamo arrivati l'altra volta, eh? se non lo lascio sbagliare non imparerà mai a impegnarsi\ldots ma io ho paura\ldots\\
C: hai il diritto di temere per lui; ma i compiti, la responsabilità di farli, sono suoi, non tuoi. Tu hai il \index{dovere}dovere di insegnargli cosa sia la \index{responsabilità}responsabilità, e questo significa lasciargli correre il rischio di sbagliare; lui si gestisce le sue responsabilità e tu ti gestisci la tua paura. E gestire la tua paura è una cosa che hai già fatto almeno due volte, ricordi?\\
S: io? veramente la paura è una cosa che mi blocca\ldots\\
C: tu eri preoccupata della sua capacità di inserirsi, ma quando si è trattato di iscriverlo a calcio, hai esplicitamente scartato la squadra del quartiere, proprio perché non stesse sempre con gli stessi amichetti; e prima ancora, quando imparava ad andare in bici,  lo hai lasciato andare perché imparasse da solo che per stare in piedi deve pedalare. In entrambe i casi avevi paura per lui, ma hai lo stesso fatto quello che era meglio per lui, non per te. Ora, in questo caso specifico, ricordandoti di queste altre esperienze, cosa senti di dover fare?\\
S: [lunga pausa] io devo trovare il modo di dirgli che ormai sa fare da solo e non ha più bisogno sempre della mamma. Però anche che se ha bisogno io ci sono sempre!\\
C: certo, sei pur sempre sua madre: lo stai solo lasciando fare le sue esperienze, anche se temi per lui. In che modo pensi di dirglielo, esattamente?\\
S: beh, aspetto che ci siano i compiti di Italiano e poi gli dico così, ''guarda, ormai sei in prima media, nelle ultime volte ho visto che sai fare da te, non c'è bisogno che io stia sempre qui, puoi fare anche i compiti da solo\ldots quando hai bisogno puoi chiamarmi, sennò li guardiamo magari assieme alla fine\\
C: te lo ha chiesto lui di guardarli assieme alla fine?\\
S: no [ride]\\
C: pensi che sia necessario riguardarli assieme?\\
S: no, ma io mi sentirei più tranquilla\\
C: lascia che io doppi la tua ultima frase: ''ti lascio fare i compiti da solo, ma non mi fido che tu li faccia bene, quindi voglio controllarli''. Cosa ne dici?\\
S: non dovrei controllarglieli, eh?\\
C: tu cosa ne pensi?\\
S: penso che non dovrei. Voglio dire, tranne che se lui me li fa vedere, ecco\\
C: e come potresti dirgli una cosa del genere?\\
S: potrei dirgli ''se qualche volta vuoi farmi vedere cosa hai fatto mi fa piacere''\\
C: e quando pensi di dirglielo?\\
S: [pausa] lui ha di nuovo italiano dopodomani, gli darà come al solito dei compiti, glielo dico quando torna a casa, quando si mette a farli; ma io mi sentirò morire\ldots
C: hai il diritto di essere preoccupata, ma abbiamo già visto che puoi fare fronte a quella sensazione, e fare la cosa che ritieni più giusta\\
S: sì\ldots certo che non sarà facile. [pausa] però finché lo tengo nella bambagia lui non può migliorare. Sai, vorrei tanto che i figli restassero sempre piccolini\ldots [sorride] ma poi non è vero. Passata la paura è così bello vederli crescere, affrontare le cose\ldots\\
C: ecco, puoi pensare a tutte le volte che hai gestito la tua paura e poi hai avuto la soddisfazione di vedere i tuoi figli riuscire, imparare. Lo hai già fatto, vedrai che sarà più facile di quel che pensi, avrai un'altra soddisfazione.\\
S: sì, lo so, come per la mia giornata libera\ldots [sorride]
\end{verse}

\noindent Concludiamo questa prima parte del colloquio ripassando \index{obiettivo!definizione del}l'obiettivo nei tempi e nei modi, \index{obiettivo!erotizzazione del} erotizzandolo con la soddisfazione che S. proverà quando il \index{figlio!secondo}figlio farà i compiti da solo. S. inizia poi spontaneamente a parlare di come è cambiato i suo rapporto con l'ansia\index{costrutti!ansia}, che aveva costituito un tema ricorrente nei precedenti incontri.

\begin{verse}
S: anche adesso la provo, ma riesco a scioglierla, mi sento più serena \ldots ho capito che sotto sotto mi faceva comodo, mi aiutava a restare nascosta, a non affrontare la mia paura di predermi le mie \index{responsabilità}responsabilità\ldots penso di essere stata a lungo una grande sfruttatrice, nel senso che per pigrizia , paura di sbagliare, di essere giudicata magari, lasciavo agli altri la responsabilità delle scelte, anche quando avrebbero dovuto essere scelte mie\ldots tuttora mi verrebbe da farlo, ma ormai vedo il meccanismo, il copione, lo vedo prima e riesco a non finirci dentro\\
C: mi stai dicendo che ti senti più forte di prima, più capace di fare le tue scelte?\\
S: assolutamente. Non è che l'ansia sia sparita del tutto, ma non è più quella cosa incontrollabile di prima, ora che so come funziona\ldots
\end{verse}

Per consolidare i cambiamenti\index{cambiamento} dell'ultimo periodo, e ricordare a S. le proprie capacità le chiedo come si sente rispetto al recente risultato, che ha ricordato poco prima, di avere chiesto una propria ''serata libera''.

\begin{verse}
C: come va con la tua serata libera? che sensazioni provi al riguardo?\\
S: ah, quella è stata una \emph{grande conquista}, assolutamente. Proprio solo l'idea di avere un mio momento, è una cosa fantastica. Che poi non esco nemmeno tutte le volte, però comunque sono io a dire che magari non mi va\ldots\\
C: direi che hai ottenuto il risultato che desideravi, che ne dici?\\
S: puoi dirlo forte! E ancora non riesco a credere come sia stato facile, se penso a quanto ero in paranoia la sera che gliel'ho detto\ldots\\
\end{verse}

\noindent Concludiamo riassumendo l'incontro, \index{obiettivo!definizione del}l'obiettivo e ancorandolo a quello precedente, felicemente raggiunto.

\begin{verse}
C: la situazione con tuo figlio mi sembra molto simile, tu che ne pensi?\\
S: beh anche con mio \index{marito}marito, prima, avevo una grande ansia\index{ansia}, e poi l'ho superata\ldots{} sono contenta un bel po'\\
C: hai dimostrato di saper cambiare la tua situazione; allora ti va bene di dire a tuo figlio\ldots\\
S: che ormai è grande e non ha più bisogno che ci sia io per fare i compiti\ldots e anche che quando vuole, posso magari guardare come li ha fatti\ldots{} ho l'angoscia solo al pensiero\ldots\\
C: l'avevi anche per parlare con tuo \index{marito}marito, e hai visto\ldots\\
S: sì, credo che sia la stessa cosa, la stessa situazione\\
C: quando \emph{hai deciso} di parlargli, a tuo figlio?\\
S: dopodomani, quando ha italiano\\
\end{verse}