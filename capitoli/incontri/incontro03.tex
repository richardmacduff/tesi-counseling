\subsection*{Sommario}

All'arrivo S. dice nuovamente di non avere temi nuovi di cui vuole parlare. Concordiamo perciò di riprendere il filo del discorso da dove lo avevamo lasciato. Dopo una breve \index{tematizzazione}tematizzazione dei contenuti dell'incontro precedente, riprendiamo la ricerca di un livello intermedio fra ''subire'' e ''imporsi'', che permetta a S. di esprimere in modo assertivo le proprie esigenze, rompendo il gioco copionico con il \index{marito!gioco copionico con}marito che abbiamo identificato nel precedente incontro.

Una volta accertata l'esistenza di questo livello intermedio, e la volontà di S. di percorrerlo, porto il discorso sulla scelta di un contesto in cui questo possa avvenire. Chiedo quindi a S. di esprimere un bisogno che vorrebbe affermare in maniera assertiva\index{assertivo!comportamento}. S. decide di voler ottenere del tempo per sé. Definiamo quindi questo come \index{obiettivo!definizione}obiettivo, ci accertiamo che sia ben formato, indichiamo un intervallo di tempo in cui raggiungerlo e concludiamo la seduta con l'assegnazione di un compito: S., entro il prossimo incontro, chiederà al \index{marito}marito del tempo per sé, e lo farà in maniera assertiva, per spezzare il gioco copionico.


\subsection*{Frasi salienti}

Investigando l'esistenza per S. di una alternativa rispetto alla \index{polarità} polarità \emph{elemosinare/imporsi}, S. gradualmente ristruttura la propria visione polare ammettendo l'esistenza di un livello intermedio, nel quale lei è in grado di esprimere i propri desideri senza arrivare a un conflitto e allo stesso tempo senza rinunciare a priori alle proprie richieste. 

Con l'occasione, S. individua anche una propria esigenza che da molto tempo rimane insoddisfatta, e che potrebbe costituire il problema da cui S. generalizza la propria convinzione di essere ''incapace'' e ''inadeguata'' nei rapporti.

\begin{verse}
C: \ldots{}prima hai detto che ti sembra di dover elemosinare; ora parli di importi. Sai immaginare un livello intermedio, dove non elemosini ma dove non ti imponi?\\
S: beh sì, dovrei dire quello che voglio fare e parlarne\\
C: per esempio di che cosa vorresti parlare senza problemi?\\
S: ah beh, del fatto che ho bisogno di uscire qualche volta\\
C: in che senso uscire?\\
S: nel senso che una sera ogni tanto vorrei vedermi con un'amica, o andare a fare spese, o al cinema, anche solo a fare una passeggiata, avere del tempo per me che  fra il lavoro, porta e riprendi i bambini, i mestieri, da mangiare e il resto, non ho mai un minuto per me da sola\\
C: e ti piacerebbe \emph{prenderti un tuo spazio}, solo per te?\\
S: ah, da morire!\\
C: c'è qualcosa che ti impedisce di farlo?\\
S: [pausa]
\end{verse}

\noindent A questo punto, \index{rispecchiamento}\index{ricalco-guida}\index{empatia!corporea} S. smette di parlare per qualche minuto. Dapprima, da appoggiata che era allo schienale, assume una posizione più rigida, in cima alla sedia, con i gomiti piantati; comincia a tormentarsi le mani e a respirare in modo un po' più accelerato. Inizialmente rispecchio la sua posizione in cima alla sedia, ma poi volutamente rispondo ai suoi movimenti con movimenti appena più lenti, e misurati; dapprima in modo impercettibile, poi sempre più evidente. Infine con un gran respiro mi appoggiato di nuovo allo schienale. Dopo qualche secondo, S. fa anche lei un gran respiro, si appoggia e riprende a parlare.\marginnote{Nel corso di questa parte del dialogo, in tutte le parafrasi cambio la parola \emph{chiedere} con la parola \emph{dire}; un piccolo \emph{ricalco-guida verbale} \index{ricalco-guida} per favorire la presa di coscienza della differenza fra un permesso concesso e un accordo fra pari; nel giro di poco, S. comincia anche lei a esprimersi nei termini di \emph{dire} anziché di \emph{chiedere}.}

\begin{verse}
S: quello che mi impedisce di farlo è il timore che poi finiamo come al solito\\
C: cosa intendi con ''come al solito''?\\
S: che lui mi dice ''sì, sì, vai, tanto\dots'' e a quel punto io sbotto, lo mando affanculo e non esco\\
C: tu con quali parole glielo chiedi? Puoi farmi un esempio?\\
S: beh, intanto gli chiedo se deve uscire\\
C: sì\dots\\
S: poi lui mi chiede perché, e a me già mi viene l'ansia\index{ansia}\dots poi gli dico che se non deve uscire magari potrei uscire io\dots\\
C: questo è il ''gioco'' di cui abbiamo parlato la volta scorsa, ricordi?\\
S: sì, assolutamente, ma vedi che ci ricasco\ldots\\
C: non è detto che debba succedere, abbiamo visto che c'è un meccanismo preciso che puoi interrompere; dimmi, come ti senti quando fai questo genere di richieste a tuo \index{marito}marito?\\
S: molto ansiosa\\
C: e quando devi \emph{dirgli} qualcosa, che so, ''abbiamo finito il latte, vai a prenderne un litro'', gli chiedi sempre prima se ha altri programmi?\\
S: no, in quel caso no. Se ha altri programmi me lo dirà lui\\
C: invece quando si tratta di chiedere qualcosa che riguarda te\dots\\
S: è vero, in quel caso è come se io gli chiedessi il permesso\\
C: credi di dovergli chiedere il permesso per uscire?\\
S: beh no, non sono mica sua \index{figlia}figlia\\
C: infatti, sei sua moglie. Ma allora, non ti sembra che gli \emph{chiedi} le cose come se volessi il suo permesso?\\
S: beh sì, è proprio così. E mi dà fastidio dovergli chiedere il permesso\\
C: come potresti fare a \emph{dirgli} che vuoi uscire senza che suoni come chiedergli il permesso?\\
S: beh, potrei dirglielo con un certo anticipo, così magari non ha già fatto altri programmi\\
C: e in che modo lo diresti, con quali parole?\\
\end{verse}

\noindent Qui S. fa una lunga pausa. \index{rispecchiamento}\index{ricalco-guida}\index{empatia!corporea} Assume di nuovo la posizione ''in punta di sedia'', ma io rimango appoggiato allo schienale, come se fossi del tutto rilassato e privo di fretta, senza fare movimenti o cenni che possano sollecitare una qualche risposta alla domanda in sospeso. Restiamo in silenzio per alcuni minuti, nel corso dei quali S. torna a una posizione più radicata sulla sedia e a movimenti più lenti. Poi riprende a parlare:

\begin{verse}
S: direi ''guarda, la sera tale vorrei prendermela per me, voglio uscire con le mie amiche''. Magari potrei dire che mi prendi una sera la settimana, come fa lui con gli amici?\\
C: lo stai chiedendo a me?\\
S: [ride] o madonna lo vedi, se non me lo facevi notare ci ricasco ancora; no  no che non lo sto chiedendo a te se posso, quello che voglio dire è che vorrei chiedere a mio \index{marito}marito del tempo per me, ma ''chiedere'' nel senso di dirglielo, non nel senso di avere il suo permesso, io \ldots{} io non voglio più avere bisogno del suo permesso\\
C: quindi ti senti di potere chiedere una cosa del genere\ldots\\
S: beh direi di sì, sì certo che posso; gli direi ''senti, tu il venerdì esci con gli amici, io voglio prendermi la sera del giovedì per me''. Ecco, gli direi così\\
C: bene; come ti farebbe sentire esprimerti in questo modo?\\
S: se ci penso, non mi sento in ansia\index{ansia}. Mi fà un po' strano perché è una cosa che non ho mai fatto, ma non mi dà l'angoscia.\\
\end{verse}

\noindent Ora provo a vedere se S. desidera, e si sente in grado di, passare dal discorso ipotetico alla definizione di un \index{obiettivo!definizione del}obiettivo reale, e alla buona formazione di questo obiettivo. In quest'ultima parte \index{ricalco-guida}, uso sempre il verbo ''dire'' (che non presuppone una concessione) al posto del verbo ''chiedere'', che invece S. usa ancora.

\begin{verse}
C: cosa ne pensi, ti piacerebbe darti proprio questa cosa come primo obiettivo?\\
S: ah, molto. Io non esco mai, ma mai mai, e mio \index{marito}marito non capisce che ho bisogno anche io di tempo per me\\
C: quindi per te è molto importante tanto l'avere del tempo per te quanto che tuo \index{marito}marito capisca?\\
S: beh\ldots io vorrei che lui capisse che non lo faccio per egoismo\ldots\\
C: e se per caso lui non capisce? Che cosa è veramente importante per te?\\
S: [lunga pausa] se lui non capisce\ldots lo capirà più avanti, io così non posso continuare\\
C: bene, allora vuoi \emph{dirgli} che vuoi del tempo per te?\\ 
S: sì\\
C: se ti prendi questo tempo per te, qualcuno ne avrà un danno?\\
S: no, i ragazzi possono stare a casa una sera con lui, è già successo altre volte\\
C: bene, quanto tempo per te vorresti, esattamente?\\
S: vorrei ritagliarmi una sera alla settimana tutta per me. Anche se non so ancora cosa ci farò tutte le volte\\
C: questo potrai deciderlo di volta in volta. In che modo pensi di dirglielo?\\
S: come abbiamo detto adesso. Con una settimana di preavviso, diciamo\\
C: e quando penseresti di farlo?\\
S: eeeeh\dots magari fra un paio di settimane?\\
C: perché non provi invece a \emph{dirlo} questa settimana per la prossima? Pensi di poterlo fare?\\
S: oddio, non ho mai detto una cosa del genere\dots\\
C: puoi semplicemente \emph{dirlo} come lo hai detto a me\\
S: ma non sono sicura di riuscirci\ldots\\
C: quello è un altro problema; per questa volta \index{obiettivo!definizione del}l'obiettivo è \emph{dire} ciò che vuoi; te la senti?\\
S: sì\\
C: che sera scegli?\\
S: mi piacerebbe il giovedì\\
C: bene, vada per il giovedì; puoi \emph{dirglielo} prima che finisca questa settimana?\\
S: beh\ldots sì\\
C: come ti sentirai dopo averglielo detto?\\
S: se ci riesco sarà bellissimo, mi sentirò molto sollevata\\
C: e se ti dice di no cosa rispondi?\\
S: [ride] lo mando affanculo? [pausa] No no\ldots ci devo pensare, gli dico che ci devo pensare\\
C: pensi di avere necessità di valutare in anticipo come reagire a una sua eventuale risposta negativa?\\
S: [pausa] beh no, non proprio. Posso sempre cavarmela dicendo che dovremo riparlarne meglio; lui non ne farà una questione\\
C:  bene, allora cosa ne dici provare a dirglielo questa settimana, come dicevi? Se poi invece non te la senti, possiamo riparlarne la prossima volta e valutare cosa potrebbe succedere nel caso che lui si opponga. Che ne dici?\\
S: ah, mi piace così\\
C: bene, allora siamo d'accordo? Ti prendi questa responsabilità?\\
S: sì.\\
\end{verse}

\noindent Chiudo l'incontro con una breve \index{tematizzazione}tematizzazione riassuntiva di quanto ci siamo detti, e quando arrivo alla consegna chiedo che S. la ripeta con parole sue, cosa che sembra darle una certa soddisfazione.

 
