\subsection*{Sommario}

All'inizio del secondo incontro chiedo a S. se desidera parlare di qualcosa di specifico a cui ha pensato o che è accaduto dall'ultimo incontro, ma risponde che non ha niente di specifico, anzi, vorrebbe riprendere dall'ansia\index{ansia}. Personalmente, ho preso nota di due obiettivi a cui dovremmo tendere già da questo secondo incontro:
\begin{enumerate}
\item cercare di definire che cosa, nel rapporto con il marito, S. vive come problematico in questo momento della propria vita
\item individuare uno stato che S. desidera raggiungere
\end{enumerate} 

\noindent per poi arrivare, in uno dei prossimi incontri, alla definizione di un \index{obiettivo}obiettivo ben formato da dare alla consulenza.

Durante l'incontro, ho occasione, tramite un ABC \index{ABC} , di chiarirmi sui motivi per cui S. si senta egoista, e di analizzare con lei un gioco Berniano\index{gioco Berniano} che sembra caratterizzare molte sue interazioni con il \index{marito}marito.


\subsection*{Frasi salienti}

\noindent S. riprende parlando della sua ansia, e cerco di sondare il \index{costrutti!ansia}costrutto con delle tecniche di laddering\index{laddering}:

\begin{verse}
C: \ldots{}puoi dirmi qualcosa di quest'ansia che provi?\\
S: la provo sempre, è sempre con me\\
C: ci sono dei casi in cui non provi ansia?\\
S: sì. se faccio qualcosa per \index{dovere}dovere, non sono ansiosa\\
C: puoi raccontarmi un esempio?\\
S: beh, quando per esempio sono sul lavoro, capita spesso che per fare la mia parte devo usare anche il lavoro che hanno fatto gli altri, ma ci sono volte che non è già tutto pronto, allora per esempio se mi serve che una collega faccia una cosa glielo chiedo, fare quello non mi dà problemi\\
C: e se invece chiedi qualcosa che non riguarda un tuo \index{dovere}dovere?\\
S: allora mi viene l'ansia perchè mi sento egoista\\
C: quindi se capisco bene: se sei tenuta, se in un certo senso è tuo \index{dovere}dovere chiedere una cosa, o un aiuto, non provi ansia; viceversa, se chiedi qualcosa per una tua libera scelta, l'ansia si manifesta?\\
S: sì\\
C: dici che l'ansia ti viene perché ti senti egoista; mi stai dicendo che pensi di essere  egoista?\\
S: penso che dovrei farcela da sola, e che se invece chiedo lo faccio perché sono egoista\\
C: non credi che ciò che chiedi ti possa spettare?\\
S: sì, ma anche se so che mi spetta \emph{tendo a sentirmi egoista}\sidenote{enfasi mia} lo stesso
\end{verse}

\noindent Qui abbiamo un ulteriore ABC\index{ABC}: S. si accorge (A) di avere bisogno di aiuto; ma (B) pensa che sia egoista chiedere aiuto e quindi (C) si sente in ansia\index{ansia}. In particolare noto che S. parla delle proprie sensazioni come se fossero propri pensieri.
Mi è sembrato opportuno suggerire qualche verbalizzazione per ricollegare questa ''ansia'' generalizzata agli altri \index{costrutti}costrutti da lei indicati a inizio seduta:

\begin{verse}
C: vuoi dire che quando fai qualcosa per tuo piacere tendi a \emph{sentirti in colpa?}\\
S: sì, io non riesco a sacrificarmi come mia madre\\
C: vuoi dire che una madre non può chiedere niente per sé? Mi parli di questo \emph{sacrificio}? che cosa significa per te?\\
S: beh, io ho sempre avuto l'esempio di mia madre che si è sacrificata in tutto e per tutto per mio padre. Qualsiasi cosa lui diceva era legge, se quello che lui chiedeva non era fatto alla perfezione lei andava in crisi\\
C: anche tu ritieni di doverti comportare così?\\
S: l'ho fatto, come moglie e come madre, ma non ero io\\
C: ricordi di una volta in cui ti sei comportata in maniera diversa?\\
S: sì, una volta mia madre ed io volevamo fare un viaggio con mio padre, ma lui traccheggiava e non si decideva mai, rischiavamo di non poter fare più i biglietti; allora io gli ho detto: se non puoi partire, andiamo da sole'', e così è stato\\
C: e come ti ha fatto sentire l'avergli detto così?\\
S: mi ha fatto sentire bene, in pace\\
C: quindi se ho capito bene in quell'occasione non hai \emph{sacrificato} la possibilità del viaggio per aspettare la decisione di tuo padre\\
S: no, l'ho affrontato\\
C: quindi hai \emph{affermato il tuo diritto} a qualcosa che facevi per \emph{piacere}?\\
S: sì, esattamente\\
C: e come ti ha fatto sentire?\\
S: \emph{mi sono sentita bene un bel po'}%
\sidenote{enfasi mia; S. pronuncia quest'ultima frase con molto trasporto}
\end{verse} 

\noindent Decido di continuare chiedendo un altro esempio di comportamento\index{assertivo!comportamento} assertivo, per renderle evidente come sia in effetti già stata in grado di avere comportamenti che negano quel suo ''non sapersi far valere'' che invece presenta come assoluto:

\begin{verse}
C: ti ricordi un altro esempio in cui ti sei comportata in una maniera che non ti ha fatto sentire di sacrificarti?\\
S: sì, quando sono stata bocciata in prima superiore\\
C: che cosa è successo quella volta?\\
S: facevo bene gli scritti, ma scena muta agli orali; mio padre l'ha presa come un affronto personale, per lui era fondamentale che tutti i suoi \index{figli}figli eccellessero a scuola. Ha smesso di parlarmi. Mi ha detto solo: ''non sei da Liceo, vai alle Magistrali''.\\
C: e tu cosa hai risposto?\\
S: quella volta ho detto no. Gli ho detto ''no, o vado al Liceo o non vado più a scuola''\\
C: e come ti sei sentita dicendolo?\\
S: molto bene. Ho continuato il Liceo, ma mio padre è diventato ancora più distante.\\
C: se ho capito bene mi stai dicendo che l'affermazione di te ha comportato l'allontanamento di tuo padre\\
S: sì, e la mia sofferenza per questo\\anche
\end{verse}

\noindent S. ritorna sul tema del rapporto con il padre e sul senso di sacrificio personale che dice esserle stato trasmesso dalla madre. Alla domanda di cosa del suo rapporto con il padre la condizioni oggi, e in quale modo, S. propone il suo rapporto con il \index{marito!rapporti con}marito, che lei dice essere ''simile'' a suo padre. Cerco allora di esplorare i punti di somiglianza e di differenza fra le due figure.

\begin{verse}
S: \dots e mio marito è simile a mio padre.\\
C: in quale modo tuo marito è simile a tuo padre?\\
S: anche con mio marito mi sento sempre in colpa, anche se è diverso\\
C: in quale modo è diverso?\\
S: mio padre era un tipo impositivo, facile alla rabbia, si doveva fare sempre quello che voleva lui o erano urlate. Invece mio marito fa la vittima, fa passare me per quella cattiva. Ma finisce che io mi sento sempre quella inadeguata. Io non so affrontare le discussioni, è sempre stato così. Perché l'aggressività espressa mi urta, mentre quella passiva mi ferisce. Mi sento sempre fuori posto\\
C: in quale modo si esprime il tuo non saper affrontare le discussioni con tuo marito?\\
S: tipo quando io gli chiedo qualcosa, qualcosa che voglio fare io, per dire, tipo che vorrei uscire, e già mi fa fatica, perché penso ''ma magari vuole uscire lui'' e lui mi risponde tipo: ''vai, vai, tanto, \dots''. Ecco, quello non lo sopporto.\\
C: cosa intendi quando dici che non lo sopporti?\\
S: che non capisco perché devo elemosinare per una cosa che piace a me\dots non sono sua \index{figlia}figlia\\
C: infatti non sei sua figlia, sei sua moglie\\
S: infatti; solo che lui mi dà sempre\emph{ l'azzica}, non prende mai una posizione, così finisce che sono io quella egoista, io mi sento inadeguata\\
C: che cosa è questa \emph{''azzica''}, un modo per  dire che lui ti provoca?\\
S: sì; lo fa sempre sembrare come se io chiedessi di continuo, ma la verità è che lui esce tutte le volte che vuole, gli amici, il calcetto, io non esco mai. E però se una volta voglio andare a fare un giro con un'amica lui me lo fa pesare come se mancassi al mio \index{dovere}dovere\\
C: tu non hai il diritto di esprimere ciò che desideri?\\
S: sì, ce l'ho, ma mi sembra di impormi\\
C: a quanto capisco, l'alternativa è fra tacere i propri bisogni o imporsi, senza temini intermedi?\\
S: sì, esatto, quando chiedo qualcosa per me mi sembra sempre di impormi\\
C: in che senso ti sembra di importi?\\
S: nel senso che quando lui mi \emph{dà l'azzica}\marginnote{\emph{dare l'azzica} (dial.): provocare}, con quel tono di sufficienza, a me mi scatta e allora dico che non voglio più uscire.
\end{verse}

\noindent Questa modalità della ''provocazione'' del \index{marito!provocazione da parte del}marito che scatena in S. la negazione per ripicca delle proprie esigenze mi ricorda un \index{gioco Berniano} gioco copionico Berniano; per valutare questa ipotesi mi confronto con S. e cerchiamo di individuare il tornaconto che entrambi traggono:

\begin{verse}
S: \ldots{}finisce che io mi sento incompresa, come al solito, io non posso mai prendermi del tempo per me\\
C: quindi possiamo dire che il tuo tornaconto è la conferma della tua posizione esistenziale, del tuo sentirti vittima?\\
S: sì, è così\\
C: e tuo \index{marito!tornaconto del}marito, che tornaconto credi che abbia da questo gioco?\\
S: lui dice che è sempre così, che lui è disponibile ma tanto a me non va mai bene\\
C: possiamo dire che lui vede confermato il suo ruolo di ''buono'' ma anche di  ''incompreso''?\\
S: oh, assolutamente!
\end{verse}

\noindent \index{tematizzazione}Tematizzo quello che mi sembra il \emph{gancio} \index{gancio} del gioco \index{gioco Berniano}Berniano, e che S. ha appena descritto:
\begin{verse}
C: perciò, quando hai un'esigenza personale chiedi a tuo marito, lui ti risponde in un certo modo, tu reagisci in un certo modo, e il meccanismo del gioco si ripete\\
S: esatto, e sto cominciando a chiedermi perché debba sempre andare così, io sono stufa di elemosinare, non sono sua \index{figlia}figlia\ldots forse è qui il problema?\\
C: lo stai chiedendo a me?\\
S: ecco, vedi? \`{E} proprio così che succede, invece di dire io una cosa che sento, cerco la conferma da te, cavolo, è proprio come faccio con lui!\\
C: me lo puoi spiegare meglio?\\
S: voglio dire, non mi basta essere io convinta di qualche cosa, io con lui e appena adesso anche con te, cerco sempre qualcuno che mi dica che ho davvero ragione, è qui l'inghippo!\\
C: allora puoi partire da qui e riflettere su come non ricadere nel gioco\\
\end{verse}

\noindent Una volta individuato e delineato il meccanismo copionico, il dialogo prosegue cercando di valutare come sia possibile non ricadere nel meccanismo. Concordiamo che sarà necessario, nel prossimo incontro, trovare un modo diverso con cui S. possa affrontare di esporre una propria richiesta nei confronti del marito. La diversità dovrà riguardare i modi con cui la richiesta viene posta e i modi in cui S. reagirà. Essendo però giunti alla fine del tempo stabilito, procedo a una \index{tematizzazione}tematizzazione riassuntiva e concordo con S. che, se non emergono nuovi temi di cui voglia parlare al prossimo incontro, potremo proseguire su questo argomento.

A pagina \pageref{fig:seduta1-2} riporto una \emph{mindmap} dei contenuti emersi nelle prime due sedute, e delle loro principali relazioni.