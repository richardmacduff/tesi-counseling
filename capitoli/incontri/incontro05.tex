\subsection*{Sommario}
All'arrivo, richiesta se ci siano elementi nuovi, S. espone alcune difficoltà contingenti nei rapporti con il \index{figlio!secondo}figlio più piccolo: lo svolgimento dei compiti, e in particolare il rapporto con l'insegnante; di nuovo, nell'esporre queste difficoltà, S. indica la propria reazione come ''ansia''\index{costrutti!ansia} e \index{senso di inadeguatezza}''senso di inadeguatezza''. Il colloquio permette di avanzare l'ipotesi, che dovà venire esplorata, che alcuni problemi che S. attribuisce al figlio, come ad esempio i rapporti con l'insegnante, possano essere in realtà proiezioni di \index{svalutazioni}svalutazioni che S. fà riguardo a se stessa.
 
\newthought{Nel corso del colloquio} cerco di esplorare i vari aspetti del rapporto di S. con il \index{figlio!secondo!rapporti con}figlio e con la di lui insegnante, individuando alcune incongruenze fra quanto S. sostiene avvenire ''sempre'' e quanto, per sua ammissione, accade. Posta di fronte a queste apparenti incongruenze, S. nota delle differenze nelle proprie valutazioni, quando queste vengono fatte in modo istintivo e quando vengono prodotte in modo più ragionato. Infatti, le sue valutazioni riguardo a: 
\begin{itemize}
\item il rapporto con l'insegnante del \index{figlio!secondo!rapporti con}figlio minore
\item il suo timore di essere manchevole nei confronti del figlio minore
\end{itemize}
sono molto diverse nei due casi; la valutazione istintiva sembra suggerire una notevole problematicità, mentre la spiegazione della situazione non risulta altrettanto problematica. Verso la conclusione della seduta, inizio a ipotizzare che una parte dell'ansia\index{ansia} denunciata da S. possa forse dipendere dalle prime richieste di autonomia del figlio.

\subsection*{Frasi salienti}

\begin{verse}
S: vorrei tornare sulla mia inadeguatezza, sull'ansia\index{ansia}, perché me le ritrovo in situazioni diverse, nel senso, mi torna sempre come punto centrale dei miei comportamenti\ldots\\ cioè me lo sento molto forte in questi giorni\\
C: è successo qualcosa di particolare di cui vuoi parlare?\\
S: beh, ho il problema della scuola di mio \index{figlio!secondo}figlio piccolo, a parte il calcio, tutte le difficoltà\ldots{}purtroppo ha incontrato una professoressa di italiano \emph{terribile}, gli dà un fracasso di compiti, ma di compiti \emph{inutili}, nel senso gli dà una pagina del libro, dove ci sono gli esercizi da riempire, no? ecco, lì basterebbe scrivere le risposte, le crocette, e invece no, lei gli fa anche ricopiare tutta la pagina, ma gli dà pagine intere\ldots{}l'altro giorno siamo stati dalle due e mezza alle sette e mezza a fare i compiti di Italiano, e poi c'era ancora Storia\ldots\\
C: cinque ore per i soli compiti di italiano?\\
S: sì, ed è la seconda volta che ci capita 'sta cosa\ldots{}e io vado subito in ansia\index{ansia}\\
C: in ansia\ldots\\
S: sì, ma guarda, un'ansia da panico, quasi\\
C: e quale motivo ti dai per quest'ansia, a cosa credi sia dovuta?\\
S: eh, è dovuta al fatto che è colpa mia se lui è capitato in quella sezione, colpa mia che non sono stata attenta magari a iscrivelro in una scuola buona\ldots{}cioè, mi sento sempre inadeguata, mi sento sempre come di non essere una buona madre, che non si è informata sui professori che c'erano, che non è stata attenta, ha lasciato un po' le cose al caso\ldots{}cioè adesso non so come aiutarlo, cioè più che stargli vicina e aiutarlo, sostenerlo\ldots{}però mi sento quest'ansia dentro, forse magari perché forse vorrei risolvere la cosa subito, ma capisco che poi in realtà non si può risolvere subito\ldots\\
C: quindi vediamo se ho seguito tutto: lui ha tanti compiti; non riuscite a farli tutti quanti\ldots\\
S: lui mi fa in crisi, naturalmente, e vado in crisi pure io\\
C: ''va in crisi'' cosa significa?\\
S: che si mette a piangere, ''io non ce la faccio più'', e '' sono stanco''\ldots{}scene un po' \emph{isterichine}\ldots\\ e io purtroppo non so come fargli capire\ldots{}che non è nemmeno giusto che un bambino di quell'età passi dodici ore sui libri\\
\end{verse}

\noindent S. racconta di avere contattato altre mamme, anch'esse con problemi analoghi,  e di avere richiesto con loro un colloquio con il Preside. Le faccio notare che ciò significa che il problema non è un problema suo personale, e lo riconosce, ma replica:

\begin{verse}
S: però io non capisco perché devo avere sempre questa morsa allo stomaco\ldots{}come una cinghia, ecco\ldots{}forse è proprio vero che ho troppa fretta di risolvere le cose, gli dò troppo peso, e invece dovrei \emph{aspettaaare, valutaaare} e poi trovare magari le risoluzioni, senza che mi venga 'sto groppo\ldots{}no?\\
C: che cosa potrebbe essere una soluzione?\\
S: di soluzioni ce ne sono tante, parlare con la professoressa, col Preside, vedere come vanno le cose, e al massimo spostarlo di classe\\
C: direi che hai non una sola, ma un ventaglio di soluzioni possibili, tu come la vedi?\label{ventaglio}\\
S: sì, ce ne sono tante in teoria, ma poi\ldots{} se ci sono tutte queste soluzioni perché io devo avere questo groppo, questa cinghia che mi strizza lo stomaco?\\ 
\end{verse}

\noindent Il colloquio prosegue cercando altre occasioni in cui questa sensazione si manifesta:i problemi di scuola del \index{figlio!secondo}figlio, oppure sul lavoro. Dal racconto di S. sembra che questa ansia\index{ansia} sia la sua reazione automatica a \emph{qualsiasi} situazione problematica, e infatti lei sostiene proprio questo:
\begin{verse}
S: sempre, quando io ho un problema ho questo \emph{peso}\ldots{}e mi disturba anche molto perché le soluzioni le vedo, non capisco perché mi pesa così
\end{verse}

\noindent A questo punto chiedo a S. di ricordare situazioni, anch'esse problematiche, in cui però questo peso non si sia manifestato. S. ci pensa un po'.

\begin{verse}
S: il peso non c'è quando il problema lo posso risolvere con una \emph{azione determinata}, in un tempo determinato, no? Tipo: ho il problema, faccio la tal cosa, risolvo, tac--tac--tac\\
C: puoi farmi qualche esempio?\\
S: ma tipo: metti che mio \index{figlio!secondo}figlio si ammala, tipo ha una brutta tosse:  sento il dottore, glielo porto, finisce lì, nessun peso, nessuna ansia\index{ansia}; o quando ho fatto l'incidente con la macchina, ho avuto sì paura, ma ansia no, ho pensato: ''porto la macchina dal meccanico, non mi sono fatta male, basta''. Forse è perché è tutto molto rapido, no, il passaggio dal problema alla soluzione\ldots\\
C: forse, però prendi l'esempio della brutta tosse, magari il dottore ti dice ''gli dia questo, lo tenga a casa magari una settimana''; o quando hai avuto l'incidente: hai portato la macchina a riparare ma sei rimasta a piedi per quindici giorni, mi dicevi. E l'insegnante di tuo \index{figlio!secondo}figlio la incontrate proprio fra una settimana. Quindi l'intervallo di tempo in cui il problema sussiste è lo stesso. Che cosa c'è di diverso in questo incontro con l'insegnante che non c'è nell'avere il bimbo malato, o la macchina dal meccanico?\\
S: [pausa] c'è\ldots{}che negli altri casi io è come se delegassi la \index{responsabilità}responsabilità\ldots{}il dottore mi dice cosa fare, il meccanico ripara\ldots{}con l'insegnante invece  io devo mettermi in relazione, spiegare, criticare, non come una critica, ma per farle capire che insomma non è giusto anche per i bambini\ldots{}ecco, io \emph{non mi reputo capace} di spiegare a questa persona che non conosco, non so poi come la prende, io in questo caso non ho nessuna via d'uscita possibile\ldots
\end{verse}

\noindent La matrice di svalutazione\index{svalutazione!matrice di} (v. pag. \pageref{tab:svalutazione}) dell'AT\index{Analisi Transazionale} aiuta ad illustrare la situazione attuale:
\begin{enumerate}
\item S. riconosce l'esistenza di segnali che indicano una situazione di problema: quindi non svaluta l'esistenza degli stimoli né l'esistenza del problema (diagonali 1 e 2)
\item si sente in grado di riconoscere ulteriori segnali di evoluzione della situazione, quindi non sta svalutando la propria possibilità di reagire agli stimoli (diagonale 3)
\item sa di poter fare fronte a determinate situazioni  problematiche (diagonale 4), quindi non sta svalutando la propria possibilità di agire sui problemi
\item per contro, S. svaluta la propria \emph{possibilità} di individuare soluzioni possibili e di agire su questo specifico problema (diagonale 5)
\item lo strumento della matrice di svalutazione ci dice anche che S. svaluta la propria capacità di mettere in atto eventuali soluzioni suggerite (diagonale 6)
\end{enumerate}

\noindent Secondo la matrice di svalutazione, non ha senso intervenire suggerendo una modalità di comportamento (diagonale 6), perché S. comunque svaluta la propria \emph{possibilità} di individuare soluzioni e la propria capacità di metterle in atto (diagonale 5). Secondo l'AT, il livello corretto a cui intervenire è proprio questo, aiutare S. a:

\begin{enumerate}
\item scoprirsi in grado di immaginare soluzioni possibili
\item scoprirsi in grado di metterle in atto
\end{enumerate}

\noindent Mostro a S. la \index{svalutazione!matrice di}matrice di svalutazione e dove si collocano le sue affermazioni di questo incontro e dei precedenti relative al nuovo problema emerso. Le mostro anche che determinate ''conseguenze'' previste dalla matrice (se si svaluta $x_ij$ si svalutano anche tutte le caselle sulla stessa diagonale e quelle sottostanti) corrispondono a ciò che lei racconta di sé. Giunti nuovamente al termine del tempo stabilito, procedo a una \index{tematizzazione}tematizzazione riassuntiva e all'indicazione di come potremo procedere nei prossimi incontri.

\begin{verse}
C: allora, questa volta mi hai parlato di un problema con tuo \index{figlio!secondo}figlio, nello specifico con la professoressa di Italiano; il problema consiste nel fatto che l'insegnante secondo te assegna troppi compiti, e tu provi ansia\index{ansia} e senso di inadeguatezza all'idea di relazionarti con lei e spiegarle\ldots\\
S: sì, esatto\\
C: e abbiamo anche visto come, dalle tue parole, uno strumento dell'AT ci dice che tu, pur riconoscendo che esistono delle soluzioni, stai svalutando la tua capacità di vederle riferite a te, e stai svalutando anche la tua capacità di agire per metterle in atto\\
S: proprio così, io non mi vedo in grado di superare questo problema\\
C: ecco, nel prossimo incontro, se vuoi, ripartiremo proprio da qui: immagineremo assieme varie modalità di comportamento e vedremo se, in fin dei conti, non possano rivelarsi positive per risolvere questo problema. Magari potresti cominciare a pensare a qualche possibilità, sia che ti sembri funzionare sia che no, e la prossima volta ne possiamo parlare. Sei d'accordo?\\
S: ok, va bene\\
\end{verse}

\subsection*{Elementi di contro-transfert}\index{controtransfert}
\begin{itemize}
\item \textsl{Cosa ho sentito}\\
Durante l'incontro precedente, ho avvertito in due occasioni l'istinto di colludere con S. 
\begin{enumerate}
\item quando S. ha menzionato la professoressa ''terribile'' ho avuto l'istinto di dire ''hai proprio ragione''
\item alla menzione del ''groppo'' e della ''cinghia che strizza'' lo stomaco, ho avvertito la stessa sensazione e ho ricordato che mi è stata e mi è familiare
\end{enumerate}

\item \textsl{Perché l'ho sentito}\\
Le mie sensazioni sono state dovute al riconoscere parte della mia personale esperienza nel racconto di S.
\begin{enumerate}
\item durante il percorso di studi mi è capitato spesso di entrare in conflitto con insegnanti che valutavo impreparati, scorretti o non abili nel gestire l'aula; per questo mi sono riconosciuto nel racconto di S. 
\item mi è capitato e tuttora talvolta mi capita di avere quel tipo di reazione fisica quando mi trovo di fronte a situazioni che mi sembrano al di là della mia portata; specialmente nel passato, in quei momenti, ho seguito istinti di evitamento o di conclusione affrettata, spesso a mio svantaggio.
\end{enumerate}

\item \textsl{La mia strategia di coping}\index{coping}\\
Sono riuscito a fare fronte a questi due istinti collusivi grazie al fatto che si sono manifestati in modo molto evidente come tali, e quindi ho potuto riconoscerli e mettere in atto in ciascun caso una modalità di \emph{coping}:
\begin{enumerate}
\item la reazione del counselor che dice ''hai proprio ragione'' è così canonicamente \index{collusione}collusiva che non ho potuto non accorgermene in tempo. Questo mi ha permesso di non seguire l'istinto e invece di portare la mia attenzione sulla situazione che S. stava descrivendo, e soprattutto di ricordare a me stesso che quella descritta non era la situazione oggettiva, ma la situazione come lei la stava vivendo; ho pertanto annotato di investigare la situazione ''compiti eccessivi'' per consentire, anche a S., di valutare il punto di vista dell'insegnante e il proprio ruolo nel problema
\item l'associazione fra sensazione corporea e istinto di \index{evitamento}evitamento fa parte del mio vissuto,  non di quello della cliente. Pertanto, ho annotato di investigare a quale tipo di pensiero lei associ questa sensazione. Ho anche ottenuto di rendere meno intensa la mia sensazione invitando la cliente a fornire maggiori dettagli sul contesto in cui questa si verifichi; l'alienità di questo contesto mi ha permesso di riguadagnare una certa distanza emotiva dal racconto.
\end{enumerate}
\end{itemize}
