\subsection{Sommario}
\label{ss:sommario}

Il lavoro si è aperto in modo piuttosto deciso, con una prima serie di \index{nominalizzzioni}nominalizzazioni che la cliente, parlando liberamente, elenca in un ordine preciso:
\begin{enumerate}
\item ansia\index{ansia}
\item \index{dovere}dovere vs. piacere
\item colpa
\item inferiorità
\item sacrificio.
\end{enumerate} 

A ciascuno stimolo del counselor, S. risponde con ulteriori aspetti negativi della propria vita: 
\begin{itemize}
\item l'apatia
\item il non sentirsi amata dal \index{marito}marito
\item l'avere sempre avuto ''poca personalità''
\item la propria anassertività con il \index{marito}marito.
\end{itemize} 

\noindent A fronte di un tale numero di temi di grande peso e rilevanza, \emph{pedino} la cliente, facendo sì che si senta ascoltata mentre esprime liberamente i propri pensieri. Inizialmente, questo significa consentire alla cliente di passare da un tema a un altro in stretta successione e, nelle sue parole, in stretta connessione anche causale: apparentemente, tutti i problemi che menziona sono correlati e ciascuno è causa del successivo. 
Per una buona ventina di minuti il flusso sembra inarrestabile, e caratterizzato dal ritorno ciclico sugli stessi temi; mi ritrovo a pensare%
\sidenote{Un esempio di \index{feedback fenomenologico} \emph{feedback fenomenologico}} che, se io mi esprimessi così, sarebbe per esprimere che la mia situazione è inestricabile, e che non ho possibilità di incidere sullo stato delle cose per modificarla a mio vantaggio. Mi chiedo se sia lo stesso anche per la cliente e annoto mentalmente di chiederglielo.

Dopo circa una ventina di minuti, nei quali non faccio mancare segnali costanti di attenzione, il ritmo del parlato diminuisce. Forse, ora che è evidente che dò peso alla sua situazione, la cliente non si sente più in \index{dovere}dovere di sottolinearne la serietà.

A questo punto inizio lentamente a ripercorrere i temi emersi e a esplicitare un filo conduttore, con \index{obiettivo}l'obiettivo di individuare quali siano, fra i tanti temi, quelli effettivamente prioritari per la cliente. Allo stesso tempo, sfrutto la sua maggiore tranquillità per raccogliere altre informazioni riguardo ai suoi \index{costrutti}costrutti e alle loro relazioni.
 
La seduta ha quindi avuto due obiettivi: l'esplorazione delle relazioni fra tutti questi sentimenti negativi e l'individuazione di un qualche problema e di qualche obiettivo tangibile, pratico, da cui S. desiderasse, o si sentisse in grado di, partire. 

\subsection{Frasi salienti}
\label{ss:frasi salienti}
Inizialmente S. associa l'ansia\index{ansia} all'impressione che sente di non poter proteggere i propri \index{figli}figli, in particolare il più \index{figlio!secondo}piccolo.

\begin{verse}
S: mi sento sempre addosso quest'ansia... me la porto addosso, io vengo da una famiglia di ansiosi, mia madre era molto ansiosa, l'ansia mi appartiene come stile di vita\\
C: puoi dirmi che cosa intendi con ''ansia''?\\
S: mah, è il timore, la paura di non riuscire a proteggere la vita degli altri,\ldots\\
C: chi sono gli altri che vorresti proteggere?\\
S: i miei \index{figli}figli, specie il più piccolo\index{figlio!secondo}\\

\end{verse}

\noindent Fra i due \index{figli}figli, la cliente individua una differenza di ''bisogno di protezione''? Chiedo spiegazione:

\begin{verse}
C: \ldots{}quanti anni ha il più piccolo?\\
S: undici\\
C: e il più grande\ldots{}\\
S: ne ha quattordici\\
C: credi che il più grande non abbia bisogno della tua protezione?\\
S: sì, ma lui è più indipendente, lo è sempre stato, ha preso da suo padre\ldots\\
C: \ldots{}e il più piccolo invece?\\
S: lui è sempre stato più mammone, più \emph{coccoloso}\ldots\\
C: quindi quando un bambino è ''coccoloso'', come dici, tu pensi che cerchi protezione?\\
S: sì, certo\\
C: e credi anche che il compito di una madre sia quello di proteggere?\\
S: una \emph{buona}\sidenote{S. enfatizza  la parola ''buona''} madre, sì
\end{verse}

\noindent  A questo punto, grazie allo strumento dell'ABC \index{ABC} dispongo di  qualche dato strutturale sul \index{costrutti!protezione}costrutto della ''protezione'':
\begin{itemize}
\item Activation: il bambino cerca il contatto
\item Belief: se il bambino cerca il contatto, è per avere protezione
\item Consequence: faccio le coccole al bambino
\end{itemize}

\noindent ma anche, ad un secondo livello, della sua connessione con il ruolo della madre come viene vissuto dalla cliente\ldots

\begin{itemize}
\item Activation: faccio le coccole al bambino
\item Belief: se faccio le coccole al bambino sono una buona madre
\item Consequence: ho un senso di sollievo perché mi sento una buona madre
\end{itemize}

\noindent Questi due ABC \index{ABC} mi permettono di capire che S. interpreta la maggiore propensione al contatto fisico da parte del \index{figlio!secondo}figlio più piccolo come ''bisogno di protezione'', e allo stesso tempo giudica la propria capacità come madre su questa ''protezione''.
Mi sembra da questo si possa dedurre che le effettive richieste ed esigenze del figlio non giochino sempre un ruolo fondamentale nel giudizio che S. dà di se stessa come madre. In particolare, mi sembra che S. derivi un giudizio negativo su sé come madre non da fatti oggettivi, ma da proprie convinzioni che, in alcuni casi, non sembrano supportate da fatti.

Fatta un po' di luce sul \index{costrutti!protezione}costrutto della ''protezione'', continuo a investigare il significato che S. attribuisce al proprio ruolo di madre.

\begin{verse}
C: che cosa intendi con ''proteggere''?\\
S: quando lui è nato, era un esserino staccato da me, e io sentivo che non potevo assolutamente proteggerlo\\
C: da che cosa avresti voluto proteggerlo?\\
S: dal suo destino, dalla sua vita\\
\end{verse}

\noindent Come già al termine del precedente spezzone, noto che il costrutto della \index{costrutti!protezione}''protezione'' sottende alcune importanti \index{doverizzazione}doverizzazioni relative al costrutto \index{costrutti!madre}''madre'', che torno a investigare.

\begin{verse}
C: quindi proteggerlo dal suo destino, dalla sua vita: è questo che pensi faccia una madre?\\
S: beh, mia madre si è sempre sacrificata in tutto, per mio padre, per la famiglia, io ci ho provato a essere come lei\ldots{}[pausa] \ldots{}ma io non sono così, come lei, non ci riesco a sacrificarmi sempre\ldots\\
C: ma questo sacrificio, in che relazione lo metti con la protezione della madre ai \index{figli}figli?\\
S: ogni volta che io o i miei fratelli avevamo un problema, a scuola, una cosa qualsiasi, mia madre era sempre lì pronta a sacrificarsi lei, ogni volta che qualcuno aveva qualcosa che non andava per lei diventava quasi un'ossessione\ldots\\
C: e quando è stato il tuo momento di essere madre\ldots\\
S: io cercavo sempre di farlo piangere il meno possibile, mi sentivo \index{responsabilità}responsabile di qualsiasi suo tipo di sofferenza
\end{verse}

\noindent Qui S. riporta la sua esperienza con la seconda maternità: un momento in cui lei si è ritrovata per un periodo senza lavoro, senza amici per via di un trasloco, e che lei definisce ''buio'', ''pieno di pensieri neri'' in cui ''vedeva tutto nero''. Di fronte alle difficoltà della nuova maternità, S. si è isolata e non ha cercato l'aiuto di nessuno. Parlando di quel periodo lo chiama ''un forte esaurimento''. 

Può darsi che si riferisca a un periodo di depressione \emph{post partum}, che esula dalle mie attuali competenze e, soprattutto, non è riconducibile al \emph{qui ed ora}. Ritorno a investigare l'esperienza della maternità per avere un quadro più completo:

\begin{verse}
C: puoi dirmi qualcosa della tua esperienza come madre?\\
S: sì, io pensavo che sarebbe stato il periodo più bello della vita, e invece [pausa] sentivo sempre come una grande agitazione\\
C: c'erano momenti in cui non ti sentivi agitata?\\
S: sì, perché io ho sempre vissuto un po' come sulle nuvole, come in fase di addormentamento\\
C: potremmo dire come un torpore, un dormiveglia?\\
S: sì, come un dormiveglia\\
C: e durante questo dormiveglia non ti sentivi agitata?\\
S: no, però vedevo solo i lati brutti della maternità, non quelli belli, e mi sentivo \index{responsabilità}responsabile della sua sofferenza, solo che io nella vita non ho mai cercato aiuto, se ho un problema mi nascondo, e quindi mi sentivo inadeguata perché non potevo proteggerlo, perché non ce la facevo sempre a sopportare. Il punto è che io non so essere \emph{vittima come mia mamma}\sidenote{enfasi mia}, non ci riesco\\
\end{verse}

\noindent Questa mi sembra una \index{polarità}polarità interessante, che cerco di mettere a fuoco.

\begin{verse}
C: vuoi dire che le alternative sono solo due: essere una vittima o non chiedere nulla?\\
S: no, ci dovrebbe essere una via di mezzo, solo che io non sono mai stata capace di chiedere aiuto\\
C: allora vediamo se riesco a seguire tutto: la tua esperienza come madre è stata segnata dall'agitazione perché non riuscivi a sopportare tutto e avresti voluto chiedere aiuto, ma non ci sei riuscita, e d'altra parte non te la senti di fare una vita di rinuncia e sacrificio come quella di tua madre?\\
S: certo, è così\\
C: e che cosa, esattamente, ti dava agitazione: l'avere bisogno di aiuto, il non riuscire a chiederlo o il non riuscire a comportarti come tua madre?\\
S: [pausa, poi con visibile nervosismo e voce tremante] mi dava agitazione di trovarmi in una situazione da cui non sapevo uscire, avrei dovuto farcela da sola ma non ce l'ho fatta\\
C: capisco, però a me sembra che i bambini li hai cresciuti, quindi in qualche modo ce l'hai fatta, no?\\
S: eh, in qualche modo sì\\
\end{verse}

\noindent Mi pare che questo argomento porti con sé una forte carica emotiva; essendo arrivati al termine del tempo previsto per l'incontro chiedo a S. se è d'accordo a tenerlo per l'incontro successivo, quando avremo il tempo necessario. Ottenuto il suo consenso, procedo a una \index{tematizzazione}tematizzazione finale e chiudo l'incontro.
