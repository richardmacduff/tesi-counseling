\subsection*{Sommario}
Poiché S. non ha portato nuove problematiche, in questo incontro abbiamo continuato ad investigare il problema dei compiti assegnati dalla professoressa di Italiano al \index{figlio!secondo}figlio più piccolo, compiti che S. ritiene ''eccessivi''.
All'inizio, per iniziare a lavorare sulle \index{svalutazioni}svalutazioni, ho chiesto a S. di immaginare cosa potrebbe dire alla professoressa ossia, ho chiesto a S. di ipotizzare possibili soluzioni, di cui conosce e ammette l'esistenza ma che non ritiene di potere e di sapere applicare.

Durante il colloquio S. ha poi ripetutamente cambiato il nome della propria sensazione da ''ansia''\index{costrutti!ansia} a \index{costrutti!paura}''paura'', e abbiamo quindi cercato di investigare a cosa questa paura sia riferita. Individuato \index{elastico} l'\emph{elastico} di una precedente esperienza di S., proseguiamo investigando se c'è qualcosa, nel comportamento attuale di S., che la metta in condizioni di sperimentare l'\emph{elastico} dell'esperienza problematica vissuta in passato. Una volta trovato un comportamento che sembra essere scatenante, valutiamo i motivi che danno origine questo comportamento; l'incontro si chiude annotando di dedicare l'incontro successivo, assente l'insorgenza di problemi contingenti, a una riflessione sulla opportunità di proseguire o di modificare questo comportamento.

\subsection*{Frasi salienti}

All'inizio del colloquio, chiedo a S. se ricorda che nel precedente incontro ha ipotizzato diversi modi di intavolare la discussione con la professoressa\sidenote{si veda pag. \pageref{ventaglio}} e la invito a parlarmene:
\begin{verse}
C: bene, allora visto che non ci sono novità oggi lavoriamo un po' sulle opzioni che hai a disposizione; ricordi che la volta scorsa hai elencato  diverse possibili soluzioni al problema?\\
S: sì, me lo ricordo, dal risolvere tutto parlando con la prof fino al cambiare classe\ldots\\
C: mi vuoi dire come potresti parlare con la professoressa, cosa potresti dirle durante l'incontro che avrete?\\
S: beh\ldots{}intanto le farei notare che stiamo parlando di bambini, sono usciti sei mesi fa dalle elementari\ldots{}lei dovrebbe non lo so, capire cosa sanno, dargli tempo di adattarsi\ldots{}e poi certo, io penso che ''il più'' debba essere fatto a scuola, ecco, poi a casa ripassi, certo, ma non così\ldots\\
C: ma secondo te, che motivo può avere l'insegnante per assegnare compiti in questa quantità che trovi eccessiva?\\
S: mah, non lo so\ldots\\
C: cosa ne pensi del fatto che la scrittura sia anche una capacità manuale, a cui allenarsi, tu cosa ne pensi?\\
S: beh sì, scrivono ancora come bambini delle elementari, ma io quando ho visto le tre pagine da ricopiare\ldots{}mi ha spaventato, ecco, ho avuto paura che questa persona non sia stabile, paura per lui\\
C: ma paura per che cosa esattamente?\\
S: [lunga pausa] mah\ldots{}credo\ldots{}non so, che non \emph{mi} [enfasi mia, NdA] regga tutti questi stress\ldots\\
C: però mi hai detto che la paura è tua, non sua\ldots\\
S: già, è vero\ldots{}sono io che ho paura\ldots{}ma perché?\\
\end{verse}

\noindent La sensazione di paura viene ripetuta più volte quando chiedo a S. di dettagliare ulteriormente la sua situazione riguardo ai compiti assegnati. Analizzo con S. come la paura non sia una reazione congrua rispetto alla situazione, e le chiedo se vede qualche analogia con qualche esperienza del suo passato.

\begin{verse}
S: [pausa] dunque\ldots{}alle elementari zero, avevo buoni insegnanti; alle medie\ldots{}quella di Italiano era proprio brava, si vedeva che aveva la passione, e ce la trasmetteva, quando ci leggeva eravamo come rapiti\ldots{}se devo proprio ricordare un'insegnante terribile, ecco, quella di danza, quella era proprio pazza secondo me, urlava, ti tirava le cose\ldots{}che poi a me quando mi urli vado in confusione, faccio tutto il contrario di quello che mi dicono, tipo lei mi urlava ''vai a destra!!!'' e io andavo in confusione e andavo a sinistra\ldots\\
\end{verse}

Con S. parliamo di come la situazione passata con l'insegnante di danza sembri costituire un \index{elastico}\emph{elastico} con la situazione attuale. In pratica, S. ha paura non dell'attuale insegnante di Italiano del \index{figlio!secondo}figlio, ma della sua vecchia insegnante di danza, e la paura è che quella si manifesti di nuovo in questa. S. riconosce che ci sono delle differenze, ma insiste che ci sono anche delle affinità. Questo mi dà l'occasione di riportare il figlio nel discorso, poiché è della sua insegnante e dei suoi compiti che stiamo parlando.

\begin{verse}
S: beh quella [l'insegnante di danza, NdA] era letteralmente una pazza, una disturbata, ci tirava addosso di tutto\ldots{}questa di Italiano non è a quei livelli, però ho saputo da una mamma che ha tirato un quaderno a un bambino che aveva sbagliato un compito\ldots\\
C: beh certo una cosa del genere, se fosse verificata, giustificherebbe i tuoi timori. Però, mi viene da dire: e tuo figlio, che cosa ne pensa di quest'insegnante?\\
S: ma guarda, questa è bella perché lui, vedi, la giustifica\ldots{}tipo l'altro giorno, con tutti quei compiti, io davo in escandescenze per quanti erano e lui mi fa ''ma mamma, ce li avrà dati per prepararci, che sabato abbiamo la verifica, no?''; ti dico, sono rimasta un attimo interdetta\\
C: e a parte i compiti, lui cosa ne pensa di questa insegnante?\\
S: a lui piace, perfino. Quando gli ho chiesto come ci si trova lui mi ha detto ''mamma, devi sentire come legge per noi, stiamo tutti lì zitti zitti''\\
C: mi ricordo che anche tu mi hai detto di aver avuto una insegnante che quando leggeva restavate come rapiti\ldots\\
S: sì, è vero, ora che me lo fai notare ha detto la stessa cosa che ho detto io, è incredibile\\
\end{verse}

\noindent Proseguiamo il colloquio con S. che confronta a lungo i propri rapporti con la vecchia insegnante di danza e, dall'altro lato, i rapporti del \index{figlio!secondo}figlio con l'attuale insegnante di Italiano. Alla fine S. riconosce spontaneamente, seguendo il filo del proprio ragionamento, che non è il figlio, ma lei stessa ad esprimere difficoltà verso l'insegnante di Italiano. Il dialogo prosegue con S. che riporta i problemi del figlio a un proprio trauma infantile, e ipotizza che sia quel trauma il vero \index{elastico}elastico che le causa queste relazioni non congruenti.

\begin{verse}
S: perché, per dire, non so come rapportarmi con lei senza magari eccedere in aggressività\\
C: ma temi di essere tu troppo aggressiva o che lo sia lei?\\
S: eh se lo fosse lei, metti che si mettesse a urlare, a dirmi ''suo figlio non capisce niente, lei non lo segue abbastanza'', sarei costretta a cambiarlo di classe, e questo per me sarebbe molto più un trauma, perché dovrei spostarlo da un ambiente dove conosce tutti a uno dove non conosce nessuno, e lui non ama queste cose, non è molto molto socievole, non si adatta molto facilmente agli ambienti nuovi\ldots\\
C: te lo ha detto lui questo?\\
S: mah, sì, ha detto, che se magari conoscesse già qualche altro bambino sarebbe , lui cerca sempre un aggancio di qualcuno che conosce, però ecco, all'inizio\ldots{}anche se poi non penso che avrebbe grossi problemi tranne all'inizio\ldots{}perché l'ha fatto, cioè questa cosa l'ho fatta: quando ha cominciato a giocare a calcio, non l'ho iscritto alla società vicino casa, perché si sarebbe ritrovato tutti  i bambini con cui andava all'asilo, da tanti anni, a scuola\ldots{}e siccome è un bambino che ha di queste difficoltà, ho detto no, proviamo a fargli cambiare e a fargli conoscere un ambiente nuovo. E quindi all'inizio ho parlato con l'allenatore, gli ho chiesto di cercare di \index{coinvolgimento}coinvolgerlo\ldots{}e alla fine si è inserito bene, però comunque è uno che ha i suoi tempi, le sue particolarità, non è quello sempre compagnone, però essendo piccolo, insomma, è normale\\
C: da quello che mi dici mi sembra un bambino con delle risorse, comunque\\
S: sì, perché anche lui magari ha più paura nel pensarlo che nel farlo [pausa] beh,\emph{ io in questo\ldots{}purtroppo\ldots} [enfasi mia]\\
C: tu hai traslocato spesso da bambina?\\
S: ehh\ldots{}io ho traslocato una volta e\ldots{}è stato un trauma per me\\
C: quanti anni avevi?\\
S: avevo undici anni, da S. Maria di Loreto sono passata a\ldots{}vicino a Santa Veneranda, e a quell'epoca era tutto campi, io pensavo di avere cambiato città, addirittura, perché mia mamma non mi mandava in città con la bici, a parte che non sarei riuscita ad arrivarci, mi sarei persa\ldots{}io ci sono stata molto male, ci ho sofferto molto. Però, sai che non me lo ricordavo?\\
C: però nel caso di tuo figlio, se anche per estremo dovesse cambiare classe o scuola, non sarebbe la stessa cosa che è stata per te, lui avrebbe sempre la squadra di calcio\\
S: eh, il problema è che è più insopportabile per me che per lui\\
C: però non sei tu che devi andare a scuola\ldots\\
S: eh, ma io ho sofferto tanto, tantissimo per la scuola, perché a me non piaceva andare a scuola, e da me si aspettavano sempre tanto, chissà perché poi, e io non ho mai dato tanto\\
C: non hai una laurea in Legge?\\
S: sì, però prima di arrivarci\ldots\\
C: hai ripetuto un anno?\\
S: ripetuto un anno, poi sono sempre stata rimandata, al Liceo\ldots{}comunque li ho fatti patire, soprattutto mia mamma che viveva questa cosa con molta molta angoscia [pausa] nche lei non perché interessasse a lei direttamente, ma perché dietro c'era mio padre che pretendeva moltissimo, lui aveva un certo ruolo, era bravissimo, e quindi anche i figli dovevano essere bravissimi\\
C: e chi lo dice?\\
S: eh, nessuno, ma in casa mia le cose stavano così, lui non aveva avuto niente, noi avevamo tutte le possibilità, dovevamo essere riconoscenti e tutto quanto\ldots{}che io poi al trauma della scuola ci ripenso spesso, a quello che mia mamma ha dovuto sopportare, e mi dico ''ma io perché devo subire questo, sopportare questo?'' Perché io alla fine non sono così: sì, mi piacerebbe che un minimo\ldots{}ma alla fine rispetto che ognuno ha la sua vita, le sue attitudini\\
C: sai, mi viene in mente Karl Popper, il filosofo, lui ha scritto un libro che si intitola ''Tutta la vita è risolvere problemi''\ldots\\
S: ma certo! Lo condivido pienamente, è quello che mi dico sempre, ''un giorno c'è questo, un altro c'è quello, convivi con i tuoi problemi ma non ti far uccidere dai tuoi problemi'', anche perché non sono problemi gravi\ldots{}e se fosse un residuo che mi porto dietro, un falso problema?\\
\end{verse}

\noindent Il tempo è ormai concluso. Ricorro a una \index{tematizzazione}tematizzazione riassuntiva per mettere un punto fermo alla discussione con il quale ripartire la volta successiva.

\begin{verse}
C: lascia che provi a riassumere le cose che abbiamo scoperto fino ad ora, vedi se ti ci ritrovi: uno, tuo \index{figlio!secondo}figlio non ha un problema con questa insegnante, ti dice ''la gestisco io'', perfino gli piace, per certi versi; due, nemmeno tu hai un problema con quest'insegnante, non ci hai nemmeno ancora parlato, i compiti saranno tanti, ma in fin dei conti li avete fatti; fino a qui come va?\\
S: fin qui bene, sottoscrivo tutto\\
C: il tuo problema, invece è che questa situazione ti richiama qualcosa di precedente della tua vita, qualche problema che hai vissuto e al quale hai reagito esattamente nel modo in cui stai reagendo ora; come andiamo?\\
S: sono completamente d'accordo, ma non so quale possa essere davvero il problema del passato\\
C: io di questo ritorno a un problema del passato vorrei capire due cose: innanzitutto che cosa lo fa scattare, perché non è il tempo, non è la fretta, non è la necessità di mettersi in relazione, non è la \index{responsabilità}responsabilità che ricade su di te, non è nessuna di queste cose, lo abbiamo visto.\\
S: già\ldots\\
C: e la seconda cosa che non ho ancora capito è: qual è il tuo tornaconto. Ossia, qual è la cosa ancora più paurosa per te che, comportandoti come fai, riesci a evitare.\\
S: eh, questo non lo so, io non è che ci sto bene, così\ldots\\
\end{verse}
