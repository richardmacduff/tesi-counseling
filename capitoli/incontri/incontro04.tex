\subsection*{Sommario}
 
L'incontro si apre con la notizia del raggiungimento \index{obiettivo!raggiungimento}dell'obiettivo concordato durante l'incontro precedente: S. ha detto al \index{marito}marito che d'ora in poi prenderà per sé la serata del giovedì. S. si dice molto contenta del risultato e quasi incredula della facilità con cui lo ha raggiunto. Ha evidentemente un forte interesse a condividere questa sua esperienza, e faccio in modo che si prenda tutto il tempo e il livello di dettaglio che desidera.
Il racconto di ciò che è avvenuto, e di come S. si sia sentita prima e dopo il colloquio e di come si senta adesso occupa tutto l'incontro.

\subsection*{Frasi salienti}

\begin{verse}
C: ricordo che ci eravamo dati una consegna per oggi, no?\\
S: sì sì\ldots\\
C: vuoi dirmi come è andata?\\
S: guarda, all'inizio non volevo nemmeno dirglielo perché non sapevo come iniziare il discorso, volevo tornare qui e dirti che non ero pronta\ldots{}Poi una sera siamo entrati in argomento, e mi sono detta ''beh, adesso o mai più'' e gliel'ho detto, gli ho detto ''senti, ho deciso che il giovedì è la mia sera libera, come il venerdì per te''\\
C: e lui come ha reagito?\\
S: mah, ha avuto un attimo come se fosse stupito, poi ha detto ''ah ok, va bene'' ed è stato tutto, davvero, avremo parlato forse due minuti in tutto\ldots{}\\
\ldots\\
C: prima di parlargli come ti sentivi?\\
S: molto nervosa\ldots{}i primi due giorni pensavo di non farcela, poi di non trovare l'occasione, poi che sarebbe finita come al solito, ti dico, volevo tornare e dirti che non ero pronta\ldots{}poi però pensavo anche che in fondo non chiedevo niente di sbagliato, e soprattutto che ora che avevamo individuato il gioco ero libera di non ricaderci, quindi tutto sommato volevo tentare\ldots{}così anche se ero molto nervosa, facevo le prove da sola di quello che dovevo dire\ldots{}\\
C: possiamo dire che hai ''provato le battute''? \\
S: sì, le ho provate prima un tot di volte, solo che all'inizio mi mancava un po' il fiato, poi invece mi sono detta che non dovevo mica stare lì a fare tanti discorsi e perché e percome, e quando c'è stata l'occasione non sono stata lì a pensarci troppo, sono andata dritta al sodo e le parole mi sono venute con naturalezza\\
\ldots{}\\
C: come ti sei sentita dopo il colloquio?\\
S: beh sono andata in cucina perché non mi vedesse, mi veniva quasi da ridere dalla contentezza\ldots{}non mi sembrava vero che fosse stato così facile
\end{verse}

\noindent Il colloquio prosegue a lungo su questa linea, perché S. si esprime con molta energia; accolgo il suo racconto con interesse, perché senta accettazione e approvazione per lo sforzo effettuato e l'ottimo risultato.

Dopo che il racconto si è esaurito, chiedo a S. se abbia già usufruito di questa conquistata ''finestra di autonomia''; la richiesta del giovedì libero al \index{marito}marito risale al mercoledì, quindi S. ha già avuto un'occasione per sperimentare un proprio momento di autonomia. Mi chiedo, e le chiedo, se l'ha sfruttata.

\begin{verse}
C: hai già avuto occasione di prenderti il tuo giovedì sera?\\
S: sì, la sera dopo, che era giovedì\\
C: me ne vuoi parlare?\\
S: ma guarda, al pomeriggio mi sono sentita con un'amica, una che non vedevo da un sacco di tempo, a un certo punto mi fa ''dai, ma vediamoci una sera che facciamo una girata in città'' e a quel punto le ho detto ''ma guarda, io posso il giovedì, tu come sei messa?'' e insomma, lei era libera e quindi dopocena ci siamo viste\\
C: come ti sei sentita quando lo hai detto a tuo \index{marito}marito?\\
S: beh all'inizio titubavo un po', non sapevo come metterla, no? Poi ho pensato come ci eravamo detti, che la parte difficile era stabilire la regola, che anche io mi posso prendere una sera la settimana, che poi magari non so neanche cosa fare tutte le settimane però poi al massimo dico ''ecco, questa settimana scelgo di stare a casa'', non è una cosa che subisco, la mia sera ce l'ho, se mi va me la prendo e se no resto a casa\ldots\\
\ldots\\
S: comunque insomma, lui è arrivato, e come è andata oggi e l'ufficio\ldots{}e poi dopo i convenevoli gli ho fatto:''senti, oggi è giovedì, io uscirei con la Gianna'', gliel'ho detto così, come una cosa normale\ldots\\
C: e infatti era  una cosa normale, lo avevate concordato il giorno prima\ldots\\
S: proprio così. E infatti lui non ha battuto ciglio, ha detto ''ah sì sì, a che ora esci?'' e finita lì
C: come ti sei sentita dopo averglielo detto?\\
S: mah, benone, ti dico, è stata una cosa proprio normale, normalissima\ldots\\
C: e come ti ha fatto sentire, questa ''normalità''?\\
S: [pausa] beh guarda, in effetti ho pensato che se era così facile potevo pure pensarci prima, poi mi sono sentita\ldots{}contenta, ecco; contenta, ma proprio soddisfatta\\
C: mi parli un po' di questa contentezza?\\
S: beh, ero contenta perché intanto facevo una cosa a cui tenevo, no, uscire, avere del tempo per me\ldots{}ma poi proprio perché io non credevo che ci sarei riuscita, e men che meno senza litigate\ldots\\
C: possiamo dire che eri felice del tuo successo?\\
S: ah sì, assolutamente\\
C: e secondo te da cosa è dipeso questo successo?\\
S: mah, non era la prima volta che gli chiedevo una cosa del genere come sai\ldots{}però questa volta è stata diversa, gliel'ho detto
in modo diverso; voglio dire, che di solito iniziavo con un ''sono stanca'' oppure ''faccio sempre casa--ufficio ufficio--casa, ho bisogno di un po' d'aria'', o magari chiedendogli prima se lui aveva un impegno\ldots{}e a quel punto lui faceva sempre l'accomodante, mi diceva ''sì sì, esci pure, tanto\ldots'' e quello proprio mi mandava in bestia, guarda. A quel punto dicevo che se le cose stavano così non mi andava più di uscire, ma glielo urlavo, proprio, e lui lì con quel fare\ldots{}perché lui riusciva a far passare me per quella mezza isterica, no?\\
C: ho notato che riguardo a questa volta dici ''gliel'ho detto'', hai usato il verbo \emph{dire}; di solito usavi il verbo ''chiedere''\\
S: e infatti è proprio quello il punto! [sorride] Proprio lì sta la differenza, prima era sempre un chiedere, quasi a mendicare, no, e a me mi dava fastidio perché cosa devo chiederti, che in casa non sposti un piatto\ldots{}[pausa] Insomma, stavolta non gli ho chiesto il permesso, non ho nemmeno cercato di spiegargli che era giusto, perché insomma, \emph{è} giusto, lui si prende il suo tempo e io devo potermi prendere il mio, non è mica una ripicca!\\
C: se ti dicessi che hai avuto un comportamento assertivo\index{assertivo!comportamento} cosa ne penseresti?\\
S: penserei che è stato proprio così\\
C: hai detto che dicevi ''se le cose stavano così''; ma ''così'' come?\\
S: [lunga pausa] sì, è il gioco di cui abbiamo parlato l'altra volta, io partivo facendo la vittima e lui era il persecutore\ldots\\
C: e ti ricordi quale abbiamo concluso che fosse il tuo ''ritorno''?\\
S: che confermavo la mia convinzione di non essere OK, cioè preferivo una conferma negativa piuttosto che rischiare di fallire cercandone una positiva\\
\end{verse}

\noindent Arrivati allo scadere del tempo convenuto, sollecito la collaborazione di S. per una \index{tematizzazione}tematizzazione riassuntiva.
S. riconosce di essersi comportata in modo assertivo\index{assertivo!comportamento} e di avere potuto, grazie a questo atteggiamento, evitare di cadere nel gioco \index{gioco Berniano}Berniano che avevamo analizzato (vedi pag. \pageref{s:incontro2} e segg.). Il comportamento assertivo le ha permesso di ottenere un risultato per lei molto importante, e riconoscendo allo stesso tempo il meccanismo comunicativo in cui lei e il \index{marito}marito erano invischiati a propria insaputa. S. si dice anche felice del fatto che le cose si siano svolte in modo più semplice di ciò che si aspettava e che questo le è da stimolo per fare attenzioni in altre situazioni per riconoscere meccanismi simili. Concludiamo l'incontro senza l'assegnazione di homework.
