\subsection*{Sommario}

Il settimo incontro non si è focalizzato su un problema specifico, S. ha preferito cogliere l'occasione di due recenti verifiche scolastiche del \index{figlio!secondo}figlio minore per parlare con maggior libertà del rapporto che ha con lui e riguardo alla scuola, il che ci ha consentito di esplorare ad ampio raggio  i temi che le stanno a cuore a alcune sue motivazioni. 

L'incontro si è aperto proprio con l'annuncio dell'avvenuta verifica di Italiano, dove il ragazzo ha preso 6. La valutazione che ne dà S. è l'occasione per investigare la sua concezione del proprio ruolo di madre ed educatrice. 

\subsection*{Frasi salienti}

\begin{verse}
S: sai comunque gli sto un po' dietro perché ho paura che magari, se inizia male, poi faccia un po' fatica\ldots{}  \\
C: quanto ha preso poi in questa verifica?\\
S: Sei\ldots{}  \\
C: beh, è la sufficienza, sei basta\\
S: un ''glorioso'' sei\ldots{}   [ride] il padre mi ha detto: ''ma quanto ha preso?'' e io: ''sei'', e lui ''e tutto 'sto show per un sei?'' perché io, sinceramente, pensavo che\ldots{}   invece ce la siamo cavata\\
C: vuoi dire che se l'è cavata?\\
S: beh sì, se l'è cavata lui\ldots{}   oddio, anche lui era un po' in tensione\ldots{}   mi ha detto ''sì, non è difficile'' però era un po' agitat\ldots{}  
\end{verse}

\noindent Cerchiamo di approfondire il tema del ''figlio un po' agitato'' e S. rivela il proprio timore di essere lei la causa, o meglio l'esempio, di questa agitazione.

\begin{verse}
S: \ldots{}   adesso poi ha pure la verifica di Spagnolo\ldots{}   ora ne fanno un sacco, di verifiche poverini\ldots{}   che poi l'orale è più difficile, lui mi dice ''sai mamma, allo scritto se studi poi ci riesci, ma a parlare è più difficile\ldots{}  '' mah, se la caverà anche lui, che ti devo dire\ldots{}  \\
C: credo di sì, ci siamo passati tutti dalla prima media, in fondo\ldots{}  \\
S: sì, e non siamo morti [ride]\ldots{}   comunque dai, \emph{mi} [enfasi mia] sta passando, spero che sia una cosa permanente, perché\ldots{}   perché è brutto, vivere con l'ansia\index{ansia}\\
C: \emph{ti} sta passando?\\
S: \emph{mi}? [ride] volevo dire ''gli'' ma evidentemente\ldots{}   è proprio mio, il problema, eh?, non suo\ldots{}  \\
C: non lo so, magari il problema che vedi tu e quello che ha lui sono due cose diverse, possiamo provare a scoprirlo insieme\\
\end{verse}

\noindent S. esprime il proprio dispiacere per avere, secondo lei, ''trasmesso ai \index{figli}figli'' le proprie difficoltà di espressione. Cerco di definire con lei in cosa consistano queste difficoltà, propongo delle piccole \index{ristrutturazione cognitiva}ristrutturazioni e poi facciamo lo stesso riguardo alle supposte difficoltà dei figli.

\begin{verse}
S: \ldots{}   perché quello che mi dispiace è che gliel'ho passata io, questa difficoltà a esprimersi\ldots{}  \\
C: cosa intendi con ''difficoltà a esprimersi''?\\
S: mah, che io spesso, ho difficoltà con le parole, non solo a trovare quelle giuste, ma proprio anche a livello di significato, a volte mi manca di sapere il significato delle parole che vengono usate\ldots{}   è una cosa che è successa anche con il corso di Counseling, che spesso mi sfuggiva il significato delle parole, anche se poi questo corso mi ha fatto molto bene\ldots{}  \\
C: per cosa ti ha fatto molto bene?\\
S: per riuscire a esprimermi, a parlare, specialmente di fronte ad altre persone\ldots{}  
C: mi ricordo che hai una laurea in Giurisprudenza, vero? Pensando a una persona che fa Giurisprudenza mi viene difficile pensare che abbia difficoltà con le parole, in un settore come quello dove i termini che si scelgono devono essere assolutamente quelli, precisi, \ldots{}  \\
S: sì, ma lì non avevo difficoltà, lì è la terminologia tecnica, no, i termini sono quelli, sono definiti nel tal modo e li devi usare così, si tratta solo di impararli, mentre in altri casi è più difficile\ldots{}  \\
C: Io penso che ti esprimi con un linguaggio tutt'altro che elementare, ti faccio risentire la registrazione\ldots{}  
\end{verse}

\noindent Risentiamo alcuni minuti di registrazione e mi fermo ripetutamente a farle notare la scelta di verbi, aggettivi, che denotano una buona competenza linguista, di una persona che abbia compiuto gli studi superiori. Riprendiamo a parlare della sua distinzione fra la ''terminologia tecnica'' e ''altri casi''

\begin{verse}
C: mi fai un esempio di un caso diverso, più difficile?\\
S: beh il Counseling, per esempio, ecco\\
C: capisco. Ma anche nel Counseling c'è una terminologia tecnica che è quella, da usare in un modo preciso; si tratta solo di impararla, come dici tu\ldots{}   e a pensarci bene lo stesso discorso vale per il calcio, tuo figlio gioca a calcio, no?, e per il tuo lavoro\ldots{}   mi viene da dire che anche se parli di scarpe o di cucina ci sono dei termini precisi da usare. In un certo senso tutto il linguaggio è fatto di termini precisi, direi, tu come la vedi?\\
S: [pausa] ma sai che messa così non ci avevo mai pensato? \`{E} vero, è proprio così\ldots{}  
\end{verse}

\noindent S. riprende poi il tema di non aver trasmesso ai figli buone capacità espressive. Indaghiamo il problema fino a svelare che questa opinione di S. non trova riscontro nei fatti.

\begin{verse}
S: \ldots{}   questa capacità di parlare è una cosa che mi dispiace di avere acquisito tardi, perché a loro proprio non l'ho trasmessa
perché fra tutti e due, uno non spiccica parola, e l'altro poco ci manca\ldots{}  \\
C: puoi darmi qualche dettaglio in più?\\
S: beh quello grande, ora è in terza superiore, è sempre stato timido, quando ha cominciato il liceo i professori mi dicevano che aveva anche delle difficoltà a inserirsi\\
C: e adesso invece?\\
S: [pausa] beh adesso, oddio, non è che sia un chiacchierone, però ha i suoi amici, si è inserito bene, mi dicono che è migliorato molto\ldots{}  \\
C: gli insegnanti lo descrivono ancora con dei problemi?\\
S: no, tutt'altro, certo non è un chiacchierone, se ne sta al suo posto, non è di quelli che organizzano ''andiamo di qua, facciamo di là''\ldots{}   cioè, è uno che se qualcun altro propone magari si accoda, ma non prende sempre l'iniziativa\\
C: ci sono volte in cui prende l'iniziativa?\\
S: beh sì, ad esempio quando c'è da prepararsi per studiare, lui trova sempre da formare un gruppetto, andare a casa di uno, dell'altro\ldots{}  \\
C: e quello più piccolo?\\
S: eh, anche lui, è sempre un po' sulle sue, non isolato ecco, però non è un chiacchierone e nemmeno un trascinatore\ldots{}  \\
C: loro come ti dicono di sentirsi?\\
S: mah loro bene, adesso tutti e due hanno il loro giro di amichetti, vanno fuori, si vedono\ldots{}  \\
C: da quanto mi dici, mi sembrano due ragazzi in grado di gestire bene il rapporto con gli altri, magari potranno non essere degli estroversi, però sia scolasticamente che socialmente non sono isolati, non mancano di capacità espressive. Magari non parlano con \emph{te} quanto vorresti, ma questo è un altro problema, no?\\
S: eh sì, a me piacerebbe sempre sapere tutto quello che fanno, tutto quello che hanno in testa\ldots{}  \\
C: però la loro crescita, la loro individuazione come persone, può avvenire solo tramite un progressivo distacco da te; distacco fisico, come succede già per il più grande, che a sedici anni ha già un bel pezzo di vita che non è più sotto il tuo controllo, non fosse altro che in termini di tempo passato fuori casa; e anche distacco emotivo\ldots{}   credo che già i loro amici sappiano di loro cose che tu non sai,e mano a mano che crescono sarà sempre più così\ldots{}  \\
S: sì però a me dispiace\ldots{}  \\
C: che cosa è che ti dispiace?\ldots{}  \\
S: [lunga pausa] che si allontanano\ldots{}  \\
C: vedi, siamo arrivati a dire che il tuo timore di avergli ''trasmesso'' poche capacità espressive nasconde la tua paura che si stacchino da te\ldots{}   l'ansia\index{ansia}, in generale, è una reazione alla paura. Ti viene in mente qualcosa rispetto al problema dell'altra volta, i compiti?
\end{verse}

\noindent In quest'ultima parte riusciamo a riagganciarci al tema dell'incontro precedente, i compiti del \index{figlio!secondo}figlio minore; il dialogo avuto fino a questo punto ci permette di riprendere quel tema con una nuova prospettiva. S. ritiene di essere indispensabile per il progresso scolastico del figlio, mentre invece ciò che teme è proprio che il figlio non abbia bisogno, perché si sta affrancando da lei; è convinta che sia possibile avere cura del \index{figlio!secondo}figlio ed essere una buona madre senza riconoscere la necessità che il figlio sviluppi una propria indipendenza, anche se poi in alcuni contesti le sue azioni sono invece tese a emancipare il figlio. Colgo questa occasione per attaccare questa \index{DIB}convinzione irrazionale.

\begin{verse}
S: \ldots{}   dall'altra volta, ti dico, quell'ansia\index{ansia} c'è molto meno, poi adesso che ha fatto la verifica, certo, i compiti rimarranno sempre un problema però un problema gestibile, con la sua collaborazione\ldots{}  \\
C: mi ricordo dall'ultima volta che l'insegnante e i compiti non li viveva come un problema; anzi, l'insegnante la giustificava  e gli piaceva perfino\ldots{}  \\
S: ah no, per lui non c'è problema\\
C: bene, invece quale è invece il motivo che ti fa sentire tenuta ad affiancarlo nei compiti di Italiano, visto che non è lui che chiede il tuo aiuto?\\
S: ho paura che mi resti indietro e di non avere più la forza io di farlo recuperare\\
C: non c'è l'insegnante per questo?\\
S: sì, però ho visto che lei non è che li interroga così spesso, e lui\ldots{}   per dire, lui legge una volta, tac! ''ho studiato'', capito? Vai e vai, alla fine tocca fare duemila pagine per studiare veramente\ldots{}   e  un po' questa cosa mi preoccupa, perché poi so che\ldots{}   [lunga pausa]
C: cambio un attimo argomento: ci va andare in bicicletta, il tuo \index{figlio!secondo}figlio piccolo?\\
S: sì\\
C: a quanti anni ha imparato?\\
S: ah, tardissimo. Anche il grande\ldots{}   tutti e due tardissimo. Guarda, a G. gli ho insegnato io che lui andava in seconda media. Lui non voleva, non voleva imparare\ldots{}   Ma un anno siamo andati io e lui una settimana su mia mamma, io gli ho detto ''basta, adesso ti insegno ad andare in bicicletta''. E siccome lì non c'era nessuno che lo potesse prendere in giro, lui in un giorno ha imparato subito; io mi sono fatta un mazzo così a corrergli dietro ma lui ha imparato subito, perché non aveva paura, ''oddio non ci riesco'', si sentiva libero, e via, ha imparato. Però tardi, tutti e due hanno imparato tardi. Anche il grande, sarà stata l'estate fra la prima e la seconda elementare. Io invcece ho imparato prestissimo, a tre anni. Ma io sono sempre stata un po' spericolata\ldots{}   Oddio, dopo che ha imparato, anche lui\ldots{}   forse perché prima lo prendevano in giro perché non ci sapeva andare. Forse è per questo che ho un po' paura, perché a lui piace fare le cose, ma poi se sbaglia, se gli riescono male, poi lui è il primo a rimanerci male\ldots{}  \\
C: ma tu in quale modo puoi impedirgli di sbagliare?\\
S: guarda, io posso stargli vicino, aiutarlo a studiare, ma poi più di questo\ldots{}   anzi, sono convinta che alcune volte occorre proprio rimanerci male, perché\ldots{}    è un insegnamento di vita, devi capire da solo che certe cose vanno fatte in un certo modo, certe regole vanno seguite\ldots{}   \\
C: quando gli hai insegnato ad andare in bicicletta, di solito uno tiene la bici da dietro mentre lui pedala, hai fatto così?\\
S: sì, ho imparato così anche io\\
C: e poi cosa hai fatto?\\
S: poi quando lui aveva preso a pedalare\ldots{}    [pausa] ho corso per un po' con la bici e poi\ldots{}    poi mi sono staccata e gli gridavo ''pedala, pedala'', lui all'inizio metteva sibuto giù i piedi e si fermava, ma a forza di insistere poi ha imparato\\
C: quindi mi stai dicendo che perché lui imparasse ad andare in bici tu hai dovuto staccarti dalla bici?\\
S: [sorride] sì\ldots{}   \\
C: e cosa ti fa pensare che per imparare a studiare le cose siano diverse?
[pausa molto lunga]
\end{verse}

\noindent Il colloquio arriva, direi, alla sua conclusione naturale proprio quasi in concomitanza con i limiti temporali concordati. Negli ultimi minuti del colloquio, S. riflette sulla necessità di lasciare che il \index{figlio!secondo}figlio minore, come regola, svolga da solo i compiti, imparando a gestire i propri tempi, il proprio carico di lavoro e il proprio modo di far fronte alla fatica. S. dice di comprendere che è proprio questo che le compete come ''buona madre'', ma si dice dubbiosa delle proprie capacità di farlo.\\Uso allora alcuni \index{ricalco-guida}ricalchi-guida linguistici.

\begin{verse}
S: e se lui non ce la fa, poverino?\\
C: questo non potremo mai saperlo se non lasci che scopra da solo fino dove lo possono portare le sue forze. Come con la bicicletta: come ti sentivi, togliendo le mani dalla sella?\\
S: avevo una paura terribile che cadesse e si facesse male, anche se ero talmente presa a correre che quasi non ci pensavo. Era in quegli attimi fra quando mi fermavo io e quando si fermava lui che il cuore mi batteva forte, per la paura ma anche per l'emozione di vedere che si mette alla prova e poi la gioia riesce, tutto insieme\\
C: lui può affermarsi come persona solo se tu, di volta in volta, lasci che faccia ciò che può. E credo che le sensazioni che provi siano del tutto legittime, a patto che non te ne lasci limitare: puoi avere paura per lui senza però impedirgli di provare da solo; e puoi certamente emozionarti  nel vederlo provare, perché ognuno di noi deve evolversi da bambino a ragazzo ad adulto; ed è naturale che, come madre, i suoi successi ti diano gioia. Invece, nessuna di queste emozioni ti sarebbe disponibile se tu gli impedissi di fare da sé. Non credi che sarebbe una terribile sottovalutazione del suo potenziale?\\
S: sì, non può restare bambino per sempre, non è mica un bambolotto\ldots{}   [sorride]\\
\end{verse}

Concludo l'incontro con una \index{tematizzazione}tematizzazione riassuntiva e l'invito a riflettere su quanto ci siamo detti per poter tradurre, nei prossimi incontri, questa nuova consapevolezza in un \index{obiettivo}obiettivo concreto di cambiamento\index{cambiamento}.