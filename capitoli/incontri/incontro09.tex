\subsection*{Sommario}
In questo incontro S. riferisce di avere raggiunto il secondo \index{obiettivo!raggiungimento del}obiettivo, di come lo ha raggiunto e di come si sente al riguardo.	Il secondo obiettivo concordato consisteva nel comunicare al \index{figlio!secondo}figlio che lo riteneva in grado di eseguire fare  compiti assegnati (in particolare dalla maestra di Italiano), evitando in questo modo di venirvi \index{coinvolgimento}coinvolta.

Negli incontri precedenti avevamo fatto emergere che il ''bisogno'' di assistenza nei compiti non era del \index{figlio!secondo}figlio, ma di S. che, in questo modo, poteva evitare di affrontare la svalutazione del proprio ruolo di madre a fronte della crescente indipendenza del figlio.

Una volta raggiunto l'obiettivo, S. scopre di poter ridefinire il proprio ruolo come madre in modo da adeguarsi ed accogliere la crescente indipendenza dei figli, anziché viverla come una \index{svalutazione}svalutazione. Così facendo, S. scopre anche di essere in grado di valorizzare le potenzialità del figlio, mentre in precedenza si riteneva una ''cattiva madre'' che non aveva trasmesso ''capacità di esprimersi''.

\subsection*{Frasi salienti}

\begin{verse}
C: di cosa vuoi parlare oggi?\\

S:	ce l'ho fatta! Ho detto a [nome del figlio minore] che è abbastanza grande per fare i compiti da solo! Ti dico, avevo un groppo allo stomaco\ldots{} ma poi mi sono detta: non è lui che ha paura a fare i compiti, sei tu che hai paura che lui diventi grande! E io non voglio di certi che mio figlio venga su uno smidollato. Allora mi sono ricordata della squadra di calcio, di come si sia inserito nonostante non conoscesse nessuno, e di come si stia facendo amichetti nuovi anche qui alle medie\ldots\\
C: mi pare che tu abbia dimostrato una grande convinzione\\
S: guarda, avevo le farfalle nello stomaco, ti dico, anche perché sotto sotto avevo anche paura che lui ci restasse male, no? Così mi son fatta forza, gliel'ho detto prima che lui andasse in camera per i compiti, così, come se fosse una cosa semplice, no? Aveva appena finito di mangiare e gli faccio ''sai, ho pensato che ormai sei grande, i compiti li sai fare, non c'è più bisogno che la mamma stia lì a sorvegliarti. Gli ho detto così, ''sorvegliarti'', così lui poteva vederla come una liberazione, no?\\
C: e lui cosa ha risposto?\\
S: guarda, io mi immaginavo che dicesse ma no dai mamma, che magari frignasse un po': oh, non ha battuto ciglio! Mi guarda e mi fa: ''sì, dai, cosi me li gestisco da solo''. Ti dico, ci sono quasi rimasta male. Allora gli faccio: ''ma comunque quando vuoi che te li guardo basta che me lo dici, eh?'' e lui: ''sì, sì, ma tanto oggi sono semplici''. E si alza e va in camera. Se li è fatti senza fiatare, e senza chiamarmi! E io ero lì con un magone, ti dico\ldots\\
C: come ti sentivi?\\
S: guarda, da un lato ero contenta, no? Perché ti dico, a me mi sfiniva stare lì tutti i pomeriggi, che poi tutte le cose della casa restavano da fare\ldots{}che poi sembrava che più gli stavo dietro e più lui ci metteva, non si finiva mai\ldots{}che poi quando c'era, per dire, matematica, prima lui doveva spiegarmi come si faceva l'esercizio, e poi dovevo stare lì a  guardare che lo faceva\ldots{} ti dico, da una parte ora ho un sollievo\ldots\\
C: e come è andata la sua prima volta da solo con i compiti?\\
S: ah, guarda: lo hai sentito, tu? Ha preso, se ne è andato nella sua stanza e dopo neanche tre ore è uscito, mi dice ''ho finito, vado a giocare da Marco''. Così, la cosa più naturale del mondo.\\
C: a te sembra che il risultato sia positivo?\\
S: guarda, io ci sono rimasta quasi male: perché da un lato, ma come: quando sto lì io ci vuole fino alle sette di sera, e da solo alle cinque hai già finito? Però sinceramente sì, io avevo paura che per lui fosse chissà quale shock, invece non ha battuto ciglio, sembrava quasi contento. Poi ti dico, quando ha detto che andava a giocare ho dovuto farmi forza per non dirgli  ''no, prima guardiamo i compiti''. Ma di quello avevamo già parlato con te, lasciarlo libero di prendersi le sue \index{responsabilità}responsabilità vuol dire proprio non stare lì sempre a controllare, ci penserà l'insegnante.\\
C: e quindi lui ha fatto i compiti e poi è andato a giocare: tu come ti sei sentita?\\
S: guarda, da una parte, un sollievo! Ma ci credi che io quasi tutti i pomeriggi li dovevo passare sui suoi compiti? Arrivavo a sera che quasi non avevo avuto il tempo di prendere un caffè, con tutte le cose in casa ancora da fare, era un disastro\ldots{} d'altra parte, è un piccolo lutto, lui non è più un bambino piccolo, non ha più bisogno di quelle cose. Ne avrà bisogno altre, che magari saranno anche più difficili da dargli\ldots{} ma non posso farci niente, è la sua vita, a me dispiace che non resti per sempre il mio cucciolotto, ma è come mi hai fatto capire le volte scorse, essere una buona madre è seguirlo e aiutarlo a diventare quello che deve diventare, non obbligarlo a rimanere un bambolotto\ldots{}\\
\end{verse}

\noindent Questa frase mi permette di riagganciarmi a un tema passato e discutere di cosa S. pensa ora del suo ruolo di madre che negli incontri precedenti, sembrava escludere la possibilità di scelte ed azioni autonome del \index{figlio!secondo}figlio.

\begin{verse}
C: ora che dici ''bambolotto'' ricordo che in uno degli incontri passati avevi detto qualcosa di simile, che ti dispiaceva di vederlo crescere perché non era più il tuo ''cucciolotto''\ldots{}\\
S: ci ho pensato, sai, dopo i nostri incontri: tutte le mamme dicono ''ah, vorrei che non crescessero mai, sono così teneri da piccolini'', ma ora capisco che un conto è dirlo, un conto è pensarlo davvero, perché come idea è orribile, non stai parlando di un essere umano, che cambia, cresce, cioè, diventa un adulto, e sì è terribile quando si staccano da te, ma è il loro destino. Certo, se penso che più diventa grande più si staccherà mi viene un magone\ldots{}\\
C: come diresti che sia cambiata la tua visione di come deve essere una ''buona madre'', adesso?\\
S: vedi, io ho sempre avuto l'esempio di mia madre, che si è sempre sacrificata per tutti, ma d'altra parte ogni piccolo problema faceva una tragedia\ldots{} noi mano a mano che crescevamo cercavamo di tenerci i problemi per noi, di risolverli da soli, magari parlare con gli altri fratelli, per non dare preoccupazioni alla mamma\ldots{} ma io non sono fatta così, io non voglio essere la madre vittima, voglio che mio figlio venga da me, se ha dei problemi, che sappia che su di me, e su suo padre, può sempre contare, anche se non è che abbiamo la bacchetta magica, certe cose non le risolvono nemmeno babbo e mamma, però ascoltarti, aiutarti a capire cosa vuoi, come puoi muoverti, quello sì, quello sempre. Io questo lo pensavo anche prima, ma la vedevo più dal punto di vista del proteggerlo dai pericoli\ldots\\
C: ricordi che una volta abbiamo parlato del fatto che volevi proteggerlo ''dalla sua vita''?\\
S: sì, quella cosa mi ha aperto gli occhi. Io per proteggerlo lo stavo soffocando, avrei voluto che rimanesse sempre piccolo perché così potevo superare la paura che avevo del fatto che crescesse, che io non potessi essere sempre lì\ldots{} Invece è come mi hai fatto capire tu, io in qualunque caso non potrò mai ''essere sempre lì'', ma già da tempo\ldots{} mi ricordo quando mi hanno chiamato dal campo di calcio che si era fatto male, io l' per esempio non c'ero, ma se anche fossi stata lì, mica posso corrergli dietro perché non cada in allenamento, no? E quante altre volte è tornato a casa pieno di lividi\ldots{} e quante cose che già  fa e io lo vengo a sapere solo dopo, magari per caso, che fa con i suoi amici\ldots{} io non posso essere sempre lì, ma posso dargli gli strumenti perché se la sbrighi da solo, adesso come adesso, secondo me una buona madre è questo che fa, dargli gli strumenti e avere il coraggio di lasciarli andare, poco per volta\ldots\\
C: direi che hai appena dimostrato di essere in grado di farlo\ldots\\
S: sì, e non credevo che ci sarei riuscita\\
C: ti senti in grado di farlo anche in futuro, quando si presenterà l'occasione?\\
S: io penso che la prima volta sia la più difficile. Ma per fortuna mi hai fatto notare che ci ero già riscita, con la bici, con il calcio\ldots{} Io lì non mi ero sentita un distacco, o magari sì, ma avevo pensato che per il suo bene andava fatto. Ho pensato la stessa cosa anche stavolta, e ci sono riuscita. Io credo che ormai riesco a distinguere cosa è un pericolo per lui e cosa è paura mia di non essere all'altezza. Certo, non è mica una passeggiata, eh?\\
\end{verse}

\noindent Complimento S. per il buon risultato raggiunto e, essendo giunti al termine del tempo stabilito, procedo a una \index{tematizzazione}tematizzazione riassuntiva.
