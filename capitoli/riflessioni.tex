\newthought{In questa mia prima istanza} di un rapporto di counseling, mi sono trovato ad affrontare contemporaneamente problemi di tipo organizzativo, di setting  e metodologici. Più volte, fra un incontro e il successivo, ho fatto ricorso ai testi per chiarire in primo luogo a me stesso il senso di quanto via via emergeva e la direzione che intendevo dare al successivo dialogo. Con il succedersi degli incontri, la mia naturale propensione alla formalizzazione ha avuto la meglio sulle difficoltà di tipo organizzativo e di setting, e ho potuto concentrarmi su aree più genuinamente problematiche, quali l'analisi in tempo reale del colloquio.

Al counselor è richiesto di essere contemporaneamente \emph{attore} e \emph{osservatore}, sdoppiamento che mi richiede una notevole concentrazione. Numerose volte, solo la rilettura ripetuta degli appunti (di volta in volta passando dal livello base dell'attore al meta-livello dell'osservatore), e il confronto con i testi mi hanno permesso di comprendere la reale portata e il potenziale significato di quanto era avvenuto durante l'incontro, e di prepararmi per utilizzare queste conoscenze nell'incontro successivo.

Il ciclo di incontri, seppure breve, ha prodotto dei risultati tangibili, a riprova che il counseling a modello integrato ha una eccellente efficacia operativa. Questo modello si è dimostrato particolarmente affine alla mia propensione empirista \index{empirismo} che privilegia l'efficacia degli strumenti operativi rispetto all'adesione a una particolare teoria. Di volta in volta, strumenti e metodi appartenenti a diverse teorie si sono dimostrati utili di per sé, e per quello che è stata la mia esperienza, trovo che l'adesione a una teoria o ad un'altra non abbia necessariamente ricadute sull'efficacia del rapporto di counseling.

\section{Limiti}\label{sec:limiti}

Ho trovato difficile riuscire ad operare contemporaneamente in tempo reale sui due livelli di attore e di osservatore. In più di un'occasione, solo il lavoro di rilettura e di analisi dell'incontro mi ha consentito di accedere in modo efficace al metalivello. Ritengo, sulla base di esperienze simili in contesti diversi%
\sidenote{L'attività di programmazione (in senso informatico) e le mie esperienze in ambienti virtuali (MUD) mi hanno già esposto ad attività cognitive che si svolgono contemporaneamente su due, tre o più livelli. Nella mia esperienza, una volta che gli strumenti operativi sono interiorizzati, l'accesso ai livelli meta diventa quasi una seconda natura}, che questa difficoltà possa diminuire con il tempo e la maggiore familiarità con gli strumenti operativi. 

Avrei voluto anche avere più dimestichezza con gli \emph{homework}, che finora non avevo avuto occasione di assegnare. Credo che fra i titoli consultati mi sarebbe stato utile anche un semplice prontuario di \emph{homework}; mi sono ritrovato spesso a pensare di non sapere che cosa avrebbe potuto costituire un buon compito da assegnare, e che tipo di risultati avrebbe potuto produrre. Ho supplito a questa mancanza attenendomi scrupolosamente a quanto emergeva dai colloqui. Compiti semplici, a volte perfino banali, che però sono stati accettati di buon grado, eseguiti senza particolari problemi e hanno prodotti risultati piccoli ma tangibili. Alla luce di questa esperienza, credo che anche un approccio operativo non particolarmente fantasioso%
\sidenote{Quando penso agli \emph{homework} il mio termine di paragone naturale è Milton Erickson, che sembrava avere una fantasia irrefrenabile al riguardo. Sono felice di avere scoperto che anche un approccio meno ''creativo'', naturalmente con problematiche più semplici, ha una sua efficacia.}%
, ma rigoroso e agganciato al contesto abbia la sua validità.

Un'ultimo limite che ho riscontrato è stata la difficoltà di individuare forme linguistiche (perfettamente logiche nel parlato comune) che nel contesto dell'incontro costituiscono violazioni linguistiche o comunque segnali di attenzione. Prima fra tutte, la confusione fra \emph{sentire} e \emph{pensare}, che si è verificata molto spesso. Anche in questo caso la rilettura e il raffronto con i testi mi hanno permesso, seppure nell'incontro successivo, di affrontare la loro corretta distinzione.

\section{Risorse}\label{sec:risorse}

Ritengo che le mie precedenti esperienze lavorative in contesti cognitivi che richiedevano l'accesso a diversi livelli \emph{meta} mi siano state di grande aiuto nel dominare la complessità che scaturisce da un incontro di counseling. Credo che, con il passare del tempo e una maggiore familiarità con gli strumenti, il passaggio in tempo reale dal livello di attore a quello di osservatore mi sarà molto più agevole, cosa peraltro già verificatasi nelle precedenti esperienze.

Trovo poi che la mia predisposizione a vedere il linguaggio come oggetto e a passare attraverso molteplici meta-livelli sia una risorsa su cui investire, dipendente come è dal grado di abitudine al contesto. Maggiore l'automaticità dei comportamenti di base, più facile la gestione contemporanea di più livelli di attenzione e di analisi.

Infine, credo che il mio approccio strettamente pragmatico al counseling, in assonanza con la linea di questa scuola, sia una grande risorsa per mantenere il rapporto di counseling in un ambito pragmatico e operativo di efficacia, evitando metodi e tematiche riservati alla psicoterapia.

