\newthought{ \`{E } prassi consolidata} della scuola stimolare i nuovi praticanti allo sviluppo del proprio \textit{metamodello}, intendendo con questo termine la personale collezione di tecniche di ciascun praticante. Questo termine va inteso nella sua accezione psicoterapeutica (dovuta alla \textsc{pnl}) di%
\begin{quote}
pattern linguistici e comportamentali utilizzati da psicoterapeuti di provata efficacia per favorire il cambiamento\index{cambiamento} nei propri pazienti\cite{magic} 
\end{quote}
e non in quella della logica formale dove indicherebbe le regole di buona formazione di un modello. Questo utilizzo arbitrario di termini tratti da discipline formali o scientifiche è purtroppo comune nella \textsc{pnl} e nel counseling. Chi scrive ritiene che il fatto possa essere ricondotto al tentativo di mutuare da quelle il rigore e lo status di ''scienza'', non diversamente da quanto avviene in economia, dove modelli matematici di inaudita complicazione vengono chiamati a fare le veci della rigorosità concettuale e dell'affidabilità predittiva della teoria, pressoché inesistenti.

Seguendo la definizione di Dilts data sopra, non ha senso parlare di \textit{un} metamodello: metamodello è l'insieme totale dei ''pattern che funzionano'', da cui ciascun terapeuta attinge la tecnica o le tecniche che intende utilizzare in un dato momento con un dato cliente. è prassi di questa scuola, però, privilegiare la praticità e la ricchezza lessicale anche a scapito di un aumento della confusione linguistica e di livelli descrittivi: nella accezione \textsc{aipac}, metamodello è anche la specifica collezione di tecniche di un praticante, sulla base della sua sensibilità della specificità del cliente e del problema portato.

\`{E} divertente notare come l'accezione \textsc{aipac} di metamodello sia deliziosamente circolare: infatti, scegliere ''fior da fiore'' le tecniche che si prediligono è \textit{essa stessa} la scelta di una specifica tecnica, e questa specifica tecnica è il metamodello della \textsc{pnl}.

Per semplicità e nel rispetto della tradizione di questa Scuola, quindi, utilizzerò comunque il termine \textit{metamodello} per riferirmi al mio personale \textit{modello} operativo, dando comunque per scontato che si tratta di una operazione a livello zero e non a livello meta.

\section{L'approccio non direttivo}
\label{sec:nondirettivo}
\index{approccio non direttivo!definizione}
Questo approccio è alla base di quasi tutte le pratiche terapeutiche di tipo psicologico e di counseling. Non mi riconosco pienamente nei suoi dettami perché ritengo che in particolari situazioni un cliente possa aver bisogno proprio di essere diretto con moderazione anziché lasciato libero di scegliere. Il compito maieutico del counselor non può esaurirsi nel lasciare la persona libera di esprimersi, deve anche spingerla a mettere a frutto le sue risorse.
Detto ciò, riconosco certamente il debito con l'approccio non direttivo nella mia pratica, in particolare per il costrutto di \index{costrutti!ascolto attivo}\textbf{ascolto attivo}:
\begin{enumerate}
\item ascoltare il contenuto, cioè cosa viene detto in termini di fatti e idee
\item sospenere ogni giudizio e valutazione sulle parole e sulla persona
\item capire le finalità, il significato emotivo di ciò di cui il cliente sta parlando
\item valutare la comunicazione non verbale: il linguaggio del corpo, il tono di voce, e soprattutto la congruenza fra la comunicazione verbale e non verbale
\item avere consapevolezza della propria comunicazione non verbale e dei propri filtri percettivi
\item entrare in empatia \index{empatia}con il cliente senza colludere (ossia mantenendo la consapevolezza di chi è il problema).
\end{enumerate}

\newthought{L'accettazione incondizionata} è un ulteriore costrutto che trovo utile, intendendo con questo il riconoscere che ogni persona è un individuo, pienamente consapevole e \index{responsabilità}responsabile di ciò che fa. Altro, naturalmente, sarebbe accettare incondizionatamente l'operato di questa persona, cosa che certamente non sottoscrivo. Questo \index{costrutti!accettazione incondizionata}costrutto è talmente connaturato alla disciplina del counseling da costituire una vera e propria \emph{stance} (atteggiamento) del counselor, trasversale a qualunque specifico indirizzo egli intenda applicare.

\section{L'empatia corporea}
\label{sec:empatia_corporea}
\index{empatia!corporea!definizione}
\newthought{L'empatia corporea} è un potente strumento di costruzione dell'empatia con il cliente,  attraverso il rispecchiamento e la riproposizione (naturalmente non smaccata) di caratteristiche che il counselor trova salienti nell'agito corporeo del cliente.

\newthought{La regolazione del respiro e del flusso verbale}, in particolare, oltre che della gestualità, forniscono al cliente segnali potenti, di \index{coinvolgimento}coinvolgimento e controllo del colloquio:

\begin{itemize}
\item [\emph{coinvolgimento:}] \index{coinvolgimento}esibendo una rispondenza fra i propri ritmi fisici esteriori e quelli del cliente, il counselor è in grado di suggerire un elevato grado di coinvolgimento e partecipazione allo stato emotivo del cliente, facilitando così lo stabilirsi di una empatia profonda
\item[\emph{controllo:}] modulando opportunamente voce, respiro e gestualità, il counselor può dare origine a una graduale disempatia controllata che, specialmente in presenza di una empatia \index{empatia} profonda precedentemente stabilita, può servire per modulare verso l'alto o verso il basso lo stato energetico del cliente, portandolo al livello più adeguato per l'espressione e la gestione delle emozioni del cliente.
\end{itemize}

\section{Il Cognitivismo}
\label{sec:cognitivismo}
\index{cognitivismo}
Il cognitivismo trae spunto dall'osservazione che le parole, al di là del loro significato lessicale, non hanno un significato \emph{operativo} universale ma il loro significato dipende fortemente dal sistema di convinzioni di ciascuno. Possiamo essere d'accordo sul modo  in cui il vocabolario definisce, per esempio, la parola ''onore'', ma difficilmente ci sarà una unanimità su cosa significhi \emph{agire} con onore. Persone diverse, in assoluta buonafede e con assoluta convinzione, potranno dare letture anche opposte dello stesso \index{costrutti!letture diverse}costrutto.

Poiché il counselor si trova a dover agire sui sistemi di credenze e di valori del cliente, è evidente che la individuazione del \emph{reale} significato di un costrutto sia fondamentale.

Per questo, dal cognitivismo mutuo la tecnica del \textbf{laddering}\index{laddering}\label{laddering}. Questa tecnica permette di ricostruire i rapporti che legano costrutti di particolare valore per il cliente e di entrare, in un certo senso, all'interno del suo sistema di costrutti, valori e credenze. La tecnica del laddering è una tecnica dialogica che mira a determinare le relazioni esistenti fra \index{costrutti!relazioni fra}costrutti che rivestono una particolare importanza per il cliente. Questo si ottiene ponendo ripetutamente delle semplici domande quali:
\begin{itemize}
\item cosa significa per te X?
\item cosa è il contrario di X?
\item cosa succede (o se non succede) X?
\item se questa cosa non fosse X, cosa sarebbe?
\item perché X è importante?
\end{itemize}


\section{L'Analisi Transazionale}
\label{sec:at}
\index{Analisi Transazionale}
Sono rimasto molto affascinato dai modelli proposti dalla Analisi Transazionale\cite{AT}, dall'ampiezza della loro capacità descrittiva e dal livello di \emph{insight} che consentono i suoi strumenti, dalle ''leggi della comunicazione'', al modello dell'Io, alle indicazioni sul copione di vita.
A fronte di una enorme capacità descrittiva, però, trovo una scarsità di strumenti pratici, specialmente quando si tratti di operare su problematiche ''leggere''.
Il modello degli \textbf{stati dell'Io} (Genitore/Adulto/Bambino)%
\sidenote{%
Secondo l'Analisi Transazionale, l'Io è composto da tre \emph{stati}, che non sono astrazioni ma realtà fenomeniche e comportamentali che si estrinsecano attraverso \emph{comportamenti, pensieri e azioni} che possono essere, a seconda dello stato:
\begin{description}
\item[Io Genitore (G)] copiati dal genitore o dalle figure genitoriali
\item[Io Adulto(A)] una risposta diretta al qui-e-ora
\item[Io Bambino (B)] riproposti dall'infanzia  
\end{description}
}
 è concettualmente molto potente ma la sua applicabilità mi sembra ristretta al livello dialogico. Di certo, la disponibilità di un modello come gli stati dell'Io, capace tanto di descrivere i comportamenti (nel caso del modello funzionale) quanto di catalogare i ricordi e le esperienze risulta di grande utilità nello strutturare le informazioni, le emozioni, le esperienze e i problemi riportati dal cliente.

\newthought{Gli strumenti} che ho mutuato dall'Analisi Transazionale sono numerosi, ed operano sia a livello generale, fornendomi un quadro concettuale nel quale muovermi, sia a livello immediatamente operativo. Uno strumento molto utile a livello concettuale è il \textbf{copione di vita}, in quanto ha ottime capacità di stimolare il cliente a una riflessione e, auspicabilmente, a una profonda revisione del proprio vissuto. Constato che come \index{costrutto}costrutto non mi sembra suggerire altri utilizzi che non siano squisitamente dialogici, cosa che peraltro è pienamente in linea con le mie convinzioni personali e il mio modello operativo.

Trovo utile anche il concetto di \index{svalutazione}\textbf{svalutazione}%
\sidenote{%
La mnemonica ''AhIAhI'' è mia. I comportamenti indicati sono:
\begin{description}
\item [Astensione] svaluto la mia capacità di pensare
\item [Iperadattamento] svaluto la mia capacità di agire secondo opzioni personali
\item [Agitazione] non sento che sto pensando
\item [Incapacità] svaluto la mia capacità di risolvere il problema.
\end{description}
}%
 con cui una persona svaluta differenti livelli dell'Io e la conseguente individuazione dei comportamenti ''AhIAhI''.
Sempre per quanto riguarda le capacità descrittive e dialogiche, l'AT consente anche di catalogare gli effetti delle svalutazioni tramite l'applicazione della \index{svalutazione!matrice di}\textbf{matrice di svalutazione} (una matrice Livelli$\times$Tipi). Il tratto più interessante, e fonte di riflessione, della matrice è il suo mettere in evidenza che le svalutazioni non sono tutte uguali: alcune sono \emph{più svalutanti}, ossia ne sussumono altre.

\begin{table}[ht]
  \centering
  \begin{tabular}{c c c c c}
        \toprule
	\multicolumn{5}{c}{\textsc{Cosa si svaluta}}\\ 
    &	& Stimoli & Problemi & Opzioni \\
    \midrule
	\multirow{4}{*}{\begin{sideways}\textsc{Rispetto a}\end{sideways}}    
    &	Esistenza	& 1 & 2 & 3\\
    &	Importanza	& 2 & 3 & 4\\
    &	Possibilità	& 3 & 4 & 5\\
    &	Capacità	& 4 & 5 & 6\\
	\bottomrule
  \end{tabular}
  \caption{la Matrice di svalutazione}
  \label{tab:svalutazione}
\end{table}

\noindent Lo scopo della matrice è duplice; essa permette al counselor di:
\begin{enumerate}
\item \emph{gerarchizzare le svalutazioni}\index{svalutazioni!gerarchia di} una volta che si osserva una svalutazione, la sua collocazione sulla matrice dice al counselor quali \emph{altri} tipi di svalutazioni sono sicuramente messe in atto dal cliente
\item \emph{determinare la svalutazione principale}\index{svalutazione!principale} se si osserva una specifica svalutazione, si può concludere che il cilente metterà in atto \emph{anche} tutte le svalutazioni di grado pari o inferiore; sulla matrice, queste compaiono al disotto e alla destra della svalutazione osservata; formalmente, se la svalutazione osservata è $S_{\overline{i}\overline{j}}$ le svalutazioni che si possono dare per scontate sono $S_{ij} \mid [i\geq\overline{i}, j\geq\overline{j}]$
\end{enumerate}

\noindent A partire da una svalutazione osservata, la matrice di svalutazione permette di investigare se il cliente metta in atto anche svalutazioni di livello più generale. L'indicazione operativa è di avviare l'intervento operativo sulla svalutazione più generale, e non su quelle di ordine inferiore. Il motivo è che se l'intervento inizia da una svalutazione di livello inferiore, la co-esistenza delle svalutazioni più generali indurrà il cliente a \emph{svalutare l'intervento}. Un esempio può servire:
\begin{quotation}
\ldots è consapevole di avere una tosse stizzosa, ma non considera questo importante per lui. Non percepisce che è un problema. Nei termini della matrice di svalutazione sta svalutando l'importanza dello stimolo e l'esistenza del problema. [La diagonale di numeri 2 nella tabella \ref{tab:svalutazione}, NdR]
\`{E} ovvio allora che svaluterà qualsiasi importanza di ciò che gli avete appena detto. Perché dovrebbe fare un investimento nello smettere di fumare quanto, per quanto ne è consapevole, la sua tosse non è un problema?
\end{quotation}

\noindent La matrice di svalutazione ha una certa macchinosità, ma permette di avviare l'intervento con sicurezza sulle svalutazioni primarie, dove l'intervento può avere buoni risultati, anziché sulle secondarie, dove anche l'intervento stesso viene svalutato.

Infine, non posso non menzionare come il concetto di \index{gioco Berniano}\textbf{gioco \index{coinvolgimento}erniano} sia illuminante nel consentire di riportare comportamenti apparentemente incomprensibili alle loro cause profonde. Nonostante l'enorme numero di giochi che Berne e altri hanno già catalogato, trovo che il disporre dell strumento analitico del gioco mi sarà di grande utilità nel proporre una mia interpretazione dei problemi che il cliente mi proporrà.

Queste riflessioni (o perlomeno la mia attuale limitata capacità operative nell'ambito dell'AT) mi fanno ritenere l'Analisi Transazionale particolarmente indicata con clienti che necessitano di un approccio principalmente linguistico-astratto, e meno in quei casi in cui sia necessario ''sbloccare'' una situazione emotiva o portare alla superficie conflitti e problemi non ancora pienamente simbolizzati.

Sono fortunatamente a mio agio con strumenti di tipo linguistico, e l'AT ne fornisce una grande quantità, fra i quali, oltre a quelli già indicati, cerco di utilizzare i costrutti di \index{costrutti!posizione di vita}\textbf{posizioni di vita} e di \index{costrutti!modello contrattuale}\textbf{modello contrattuale}.

\section{La Gestalt}
\label{sec:gestalt}
\index{Gestalt}
Ammetto di sentirmi poco affine alla terapia della {Gestalt} e alla sua attenzione estrema per l'espressione delle emozioni nel \emph{qui e ora}. Ciononostante, trovo empiricamente \index{empirismo} utile la tecnica ''della \textbf{sedia calda} e della sedia vuota'' per quattro motivi fondamentali:
\begin{enumerate}
\item il cliente si può \emph{disidentificare}, ossia abbandonare momentaneamente il proprio stato emotivo e il proprio punto di vista
\item il cliente può \emph{esplorare} in modo sicuro (in quanto all'interno di una \emph{finzione scenica}) la situazione dai punti di vista degli altri attori coinvolti\index{coinvolgimento}
\item il cliente vede da una prospettiva esterna anche la propria condizione, acquisendo consapevolezza di come gli altri vedono la sua situazione
\item ogni differente lettura della stessa situazione diventa immediatamente una alternativa cognitiva e comportamentale che il cliente può consapevolmente scegliere in sostituzione della sua attuale.
\end{enumerate}

\noindent Della sedia apprezzo il valore maieutico, e il valore maieutico è una delle due caratteristiche che permettono di differenziare il percorso di counseling da altre pratiche di tipo mantico \index{discipline mantiche} (chiromanzia, cartomanzia, ecc.) che si fondano più sul dare al cliente qualcosa a cui vuole credere anziché permettergli di esplicitare ciò che sente. La seconda caratteristica che distingue il counseling da altre pratiche di tipo mantico è il tema della \index{responsabilità}\emph{responsabilità}: mentre il counseling richiede al cliente una preventiva, deliberata, esplicita assunzione di responsabilità nei confronti del proprio cambiamento per poter avviare la pratica di consulenza, le altre discipline mantiche si pongono, all'opposto, come \emph{scarichi di responsabilità}, momenti fideistici nei quali il cliente affida il proprio cambiamento\index{cambiamento} a qualcosa che è altro da sé. 

\newthought{L'amplificazione}\index{amplificazione} è un ulteriore strumento gestaltico utile nel caso che il cliente abbia difficoltà ad accedere al proprio stato emotivo, presente o passato. Di norma, quando questo avviene, un'emozione considerata non appropriata viene svalutata fino al punto di non venire riconosciuta, e di riemergere sotto forma di espressione corporea involontaria.
La tecnica consiste nel cogliere una particolare espressione corporea, che si suppone essere espressione di uno stato emotivo, e chiedere al cliente di \emph{amplificarla}, ossia di aumentare vistosamente i movimenti che la costituiscono, più volte, fino ad ottenere una espressione quasi parossistica. Proprio per questo suo parossismo, l'espressione corporea verrà vissuta come \emph{altro da sé}, il che è esattamente l'obiettivo della tecnica: sentendosi \emph{altro} dallo stato emotivo alla fonte dell'espressione corporea, il cliente troverà generalmente più semplice riconoscerla in modo consapevole e quindi nominarla. Il counselor, pur aiutando il cliente ad accedere alla consapevolezza dell'emozione amplificata, agisce in modo da mantenere inizialmente il distacco: la domanda-chiave è infatti espressa in modo impersonale, \emph{"che cos'è questo movimento?"}; una volta che il cliente ammetta l'esistenza dell'emozione una ulteriore domanda consentirà di accedere realmente allo stato emotivo, stavolta visto come proprio: \emph{"allora mi stai dicendo che provi X?"}.

\section{La \textsc{pnl}}
\label{sec:pnl}
\index{PNL}
Nonostante la mia avversione per il modo meccanicistico e semplicistico nel quale viene propagandata la \textsc{pnl}, mi riconosco nell'approccio originario (sostanzialmente scettico ed empirista) \index{empirismo!scettico} di questa disciplina, e in particolare con il concetto di \textit{metamodello} che la \textsc{pnl} sviluppa: \begin{quote}
pattern linguistici e comportamentali utilizzati da psicoterapeuti di provata efficacia per favorire il cambiamento nei propri pazienti\cite{magic} 
\end{quote}

\noindent Nella mia pratica intendo adottare principalmente \emph{tecniche} e non modelli teorici, avendo come riferimento teorico l'empirismo scettico e come criterio di scelta la mia convinzione che quelle tecniche possano funzionare.

\subsection{L'ancoraggio}
\label{sub:ancoraggio}
\index{ancoraggio!definizione}
L'ancoraggio è il nome che la PNL attribuisce a (traduzione mia)

\begin{quotation}
il processo tramite il quale un qualsiasi stimolo o rappresentazione (interna o esterna) viene associata a (e pertanto scatena) una risposta. Le ancore si verificano tanto in modo naturale quanto in modo intenzionale. Il concetto di ancora della PNL deriva dalla reazione stimolo-risposta pavloviana del condizionamento classico. Negli studi di Pavlov, il diapason era lo stimolo (àncora) che portava il cane a salivare.\cite{magic}
\end{quotation} 

\noindent L'ancoraggio mi permette di facilitare l'apprendimento, la memorizzazione e l'automatizzazione di particolari comportamenti o associazioni che desidero trasmettere. In particolare, intendo servirmene quando dovrò nella assegnazione di homework, per consentire al cliente di non cadere nei propri comportamenti abituali limitanti.


\subsection{Le violazioni linguistiche}
\label{sub:violazioni}
\index{violazioni linguistiche!definizione}
Le violazioni linguistiche individuano il cosiddetto \emph{metamodello} e si suddividono in:
\begin{enumerate}
\item generalizzazioni
	\begin{enumerate}
	\item quantificatori universali (mai/sempre, tutti/nessuno)
	\item operatori modali (devo, occorre, vorrei)
	\end{enumerate}
\item cancellazioni
	\begin{enumerate}
	\item semplici (non so che fare, non ne posso più)
	\item comparativo assente
	\item falsi avverbi (ovviamente, chiaramente)
	\item mancanza di referente (si dice, quelli)
	\item verbi non specificati (non mi rispetta)
	\item \index{nominalizzzioni}nominalizzazioni	
	\end{enumerate}
\item deformazioni
	\begin{enumerate}
	\item causa-effetto
	\item equivalenze complesse
	\item lettura del pensiero
	\item presupposti
	\item performativo mancante (è sbagliato/giusto)
	\end{enumerate}
\end{enumerate}

\noindent Le violazioni linguistiche permettono di individuare ed attaccare ragionamenti e convinzioni che, in senso costruttivista, possiamo chiamare irrazionali, ossia non funzionali \index{obiettivo} all'obiettivo \emph{consapevole} del cliente, ma invece legati a sue credenze limitanti.

\subsection{Il rispecchiamento}
\label{sub:rispecchiamento}
\index{rispecchiamento!definizione}
Quanto appena descritto è naturalmente ciò che in \textsc{pnl} viene chiamato \textbf{rispecchiamento} (in questo modo è stato sviluppato, per esempio, il Milton Model). Il rispecchiamento può avvenire in differenti modalità:
\begin{itemize}
\item adottando specifici pattern linguistici e comportamentali di provata efficacia in casi simili
\item assumendo pattern linguistici e comportamentali del cliente, per poter stabilire un \textit{rapport} profondo e poterlo quindi guidare. 
\end{itemize}

\noindent Il \index{rispecchiamento}rispecchiamento è una tecnica particolarmente utile perché consente di sintonizzarsi sulle specificità percettive del cliente, di esprimersi in un modo che lui riconosce come \textit{naturale}. La speranza è quella che la naturalezza del counselor nel ripresentare al cliente parte del suo comportamento fisico e verbale venga recepito al disotto del livello di coscienza, stabilendo il \emph{rapport} desiderato.
Per poter essere efficace, il \index{rispecchiamento}rispecchiamento deve avvenire su due livelli:
\begin{enumerate}
\item a livello linguistico, il counselor identifica e si allinea al canale primario del cliente (che può essere visivo, uditivo o cinestesico) e adotta modelli linguistici analoghi al canale primario del cliente. Quindi, con un cliente che ha un canale primario visivo, il counselor userà formule del tipo \emph{vedo}, \emph{mi è chiaro}, e simili; analogamente, con un cliente che esprime un canale primario cinestesico il counselor adotterà invece formule come \emph{sento che\ldots}, \emph{mi ci ritrovo}, e simili
%
\item a livello corporeo il \index{rispecchiamento}rispecchiamento consiste nell'adottare posizioni e movimenti corporei \emph{affini} (ma non necessariamente identici) a quelli del cliente. Se il cliente tiene le braccia incrociate e la schiena contro la sedia, il counselor potrà tenere le dita incrociate ed appoggiarsi anch'egli. Se il cliente gesticola, anche il counselor dovrà adottare gesti simili per ampiezza, frequenza e vigore.
\end{enumerate}

\noindent Il rispecchiamento deve essere effettuato in modo contenuto, per evitare situazioni grottesche simili al gioco dello specchio fra bambini. Sarà cura del counselor procedere gradualmente, restituendo al cliente posizioni e comportamenti \emph{simili ma non identici} nel modo più naturale possibile.
%
\subsection{La formazione dell'obiettivo}
\label{sub:obiettivo}
\index{obiettivo!formazione del!definizione}
%
Prima ancora del rispecchiamento, e probabilmente più importante, è però la tecnica della \textbf{formazione dell'obiettivo}. Questa tecnica, che in effetti costituisce il primo e fondamentale \emph{meta-pattern} della \textsc{pnl}\cite{magic}. Questo pattern stabilisce che l'obiettivo debba essere strutturato in un modo specifico, così da corrispondere ai seguenti criteri:
%
\begin{description}
\item [espresso in termini positivi] l'obiettivo deve descrivere ciò che il cliente desidera ottenere, non ciò che desidera evitare o ciò di cui vuole liberarsi
\item [sotto la propria \index{responsabilità}responsabilità] l'obiettivo deve dipendere da azioni esplicite e volontarie del cliente. Ad esempio, ''vincere alla lotteria'' sarebbe un obiettivo che non corrisponde a questo criterio
\item [definito nel tempo] \index{obiettivo!formazione del!definizione}l'obiettivo deve essere raggiungibile in un tempo determinato, e il lasso di tempo assegnato per il raggiungimento dell'obiettivo fa parte della definizione dell'obiettivo
\item [espresso in termini sensoriali] il cliente deve descrivere il modo in cui si sentirà, apparirà e si comporterà una volta raggiunto l'obiettivo 
\item [espresso per passi successivi] si devono specificare i passi intermedi che portano all'obiettivo, ciascuno di questi costituisce un sotto-obiettivo che deve essere  ''ben formato'' al pari dell'obiettivo principale
\item [tangibile] \index{obiettivo!formazione del!definizione}il raggiungimento dell'obiettivo deve essere evidente; sta al cliente indicare come è in grado di sapere se 'obiettivo è stato raggiunto e quando
\item [erotizzato] l'obiettivo deve essere \emph{desiderabile}; sta al cliente indicare quali siano le caratteristiche per lui desiderabili nel raggiungere l'obiettivo
\item [ecologico] l'obiettivo non deve porsi contro i valori profondi del cliente, non deve contrastare con altri suoi obiettivi, deve contribuire al miglioramento del suo stato e non causare danni o problemi ad altri o all'ambiente circostante.
\end{description}

\noindent Questa grande attenzione al \emph{modo} in cui l'obiettivo viene definito, più che all'obiettivo in sé, è dovuta alla necessità che il cliente acquisisca la consapevolezza che si ritiene necessaria a comprendere e ad esprimere non solo la propria motivazione verso l'obiettivo, ma anche il tempo e le risorse che gli saranno richiesti per raggiungerlo.
Nel counseling è un assioma indiscusso che un obiettivo non possa essere raggiunto senza che il cliente risponda appieno a tutti i requisiti appena elencati. Indipendentemente dalla fondatezza o meno, trovo \emph{pragmaticamente} utile attirare e mantenere l'attenzione del cliente sull'obiettivo che intende darsi.
In primo luogo, per dare il tempo di affiorare ad eventuali dubbi o incertezze e, in secondo luogo, per far sì che il cliente analizzi sotto ogni aspetto il modo in cui intende raggiungere l'obiettivo, i costi che questo comporta, le scelte che dovrà compiere. Considero il processo di counseling come lo sviluppo della consapevolezza nel, e il modo specifico con cui la \textsc{pnl} propone di approcciarsi all'obiettivo risponde perfettamente a questo sviluppo.

\section{La Rational Emotional Therapy}\label{sec:ret}
Dalla Terapia Razionale Emotiva o \textsc{ret} (Rational-Emotive Therapy)\cite{ret} mutuo due strumenti, il modello \textsc{abc} e il metodo \textsc{dib}.

\newthought{Il modello \textsc{abc}}\index{ABC!definizione} si basa su tre costrutti: \textit{Activation, Belief \textrm{e} Consequence}. L'idea alla base del modello è semplice: a partire da un evento attivante \textsc{a} , e sulla base di credenze \textsc{b} (che possono essere razionali o irrazionali, ossia adeguate o inadeguate alla realtà), la persona trae delle Conseguenze (e anch'esse possono essere razionali o irrazionali). L'interesse del modello sta nel fatto che il cliente è sempre conscio del livello \textsc{c}, su cui basa le proprie reazioni, ma non dei livelli \textsc{a} e \textsc{b}. Il modello è utile:
\begin{itemize}
\item per investigare le motivazioni che una persona adduce per i propri comportamenti
\item per valutare la adeguatezza o meno (ossia la razionalità o meno) di queste motivazioni alla realtà.
\end{itemize}

\noindent Secondo Silvestri, la \textsc{ret}
\begin{quote}
attribuisce importanza preminente al fatto che gli esseri umani tendono a pensare in modo irrazionale, ad avere convinzioni irrazionali, a usare male i principi della logica e del ragionamento, a formarsi una visione del mondo, una filosofia, o vari contraddittori elementi di filosofia, che hanno poco a che vedere con la realtà obiettiva dei fatti così come possiamo percepirla. In tale modo si procurano emozioni e stati d'animo straordinariamente spiacevoli, e mettono in atto comportamenti disadattanti, autolesivi o palesemente assurdi.
\end{quote}

\noindent Sebbene sia difficile non essere d'accordo con questa visione, mi dissocio dal giudizio di valore che essa implica, in questi termini: dire che gli esseri umani \emph{tendono a pensare in modo irrazionale} non è diverso dal dire che un corpo lasciato cadere cade \emph{molto velocemente}. In effetti, un corpo libero di cadere, cade con un'accelerazione pari a $9,8ms^{-2}$ che non è né molto né poco, ma è semplicemente il modo in cui deve cadere per il fatto che il pianeta Terra ha una certa massa. Analogamente, gli esseri umani ragionano nel modo in cui ragionano. Nel corso della loro evoluzione hanno sviluppato molteplici modalità di ragionamento, una delle quali è chiamata ''razionale'' per il fatto di essere trattabile in termini formali. Il ragionamento razionale ha indubbi vantaggi, non ultimo quello di essere difendibile, ma non va sopravvalutato né idealizzato. 
La disponibilità di un ampio arsenale di metodi di ragionamento che va dall'istinto all'intuizione al pensiero magico al pensiero razionale, è un \emph{vantaggio} degli esseri umani rispetto ad altre specie anche senzienti, e non uno svantaggio.

Si pensi alla profonda inutilità di una valutazione razionale dei rischi per chi si trovi in una situazione di estremo pericolo (come per esempio una casa in fiamme), e al vantaggio operativo di chi, anche a scapito di evidenze al contrario, svaluti i rischi e ponga tutte le sue energie nella ricerca di una soluzione per quanto apparentemente disperata. Credenze irrazionali? Assai probabile. Pensiero magico? Forse. Eppure, quando la situazione lo richiede, scommettere sul caso favorevole è la sola possibilità, anche se il caso favorevole è di uno su svariate migliaia.

In particolari situazioni, \emph{è razionale adottare credenze non razionali} per catalizzare le nostre energie e il nostro entusiasmo ed impiegarli nella soluzione di problemi che, a mente fredda, non presentano soluzione. La possiamo chiamare ''sospensione dell'incredulità'', ma è uno strumento potente a \emph{integrazione} delle nostre facoltà logico-deduttive.
Il problema non è tanto se una persona coltivi credenze irrazionali o agisca sulla base di queste. Il problema è determinare se le credenze sulle quali la persona agisce siano \emph{adeguate o meno} alla situazione in cui si trova.

Può sembrare follia scegliere razionalmente un comportamento non razionale, eppure questo è il modo in cui gli esseri umani sono costruiti. Fra le nostre molte qualità, la capacità di razionalizzare il non razionalizzabile è fra le più sviluppate: un essere umano può credere \emph{a qualunque cosa} se lo vuole, e motivare \emph{ex post} in modo apparentemente razionale qualunque scelta. Scopo del counselor dovrebbe essere perciò di valutare con il cliente se ciò in cui questi crede sia adeguato o non adeguato alle esigenze.

Gli esseri umani non sono esclusivamente\cite{predirr, blackswan} razionali. Crederlo, come credere che che i comportamenti non razionali siano ''difetti'' \emph{tout court}, è una pericolosa semplificazione. Siamo parzialmente non-razionali perché il mondo stesso (al livello della vita cosciente, non delle leggi fondamentali) è irrazionale. Non tutto ciò che accade ha un senso, e aspettarsi che il mondo ad ogni livello corrisponda alle nostre esigenze di razionalità è una posizione ideologica che non mi sento di condividere.
Una esposizione più approfondita di questa mia linea di pensiero esula dall'ambito di questa tesi e viene perciò rimandata ad altra occasione.

\newthought{Il metodo \index{DIB!definizione}\textsc{dib}}, acronimo di Disputing Irrational Beliefs (messa in discussione delle convinzioni irrazionali) è una semplice procedura con cui, una volta individuata una convinzione irrazionale non adeguata, il cilente viene stimolato a metterla in discussione. Lo scopo è che il cliente stesso, in prima persona, sperimenti la non adeguatezza della sua convinzione.
La procedura si basa sul rispondere, durante il colloquio o durante un homework, a sei domande, aventi come oggetto la convinzione irrazionale individuata:
\begin{enumerate}
\item quale convinzione voglio contestare?
\item posso sostenere razionalmente tale convinzione?
\item quali prove esistono della verità di tale convinzione?
\item quali prove esistono della falsità di tale convinzione?
\item quali sembrano le cose peggiori che potrebbero capitarmi se davvero le cose andassero come credo?
\item quali cose buone potrebbero capitarmi, o potrei fare in modo che mi capitassero, se gli eventi andassero nel modo che penso non dovrebbero andare?\cite{ret}[p.55]

\end{enumerate}

\section{Il feedback fenomenologico}
\label{s:feedback}
\index{feedback fenomenologico!definizione}
Il feedback fenomenologico, pur non essendo attribuibile a nessun modello specifico, costituisce un cardine dell'attività di counseling. Il suo scopo è di ancorare la comunicazione counselor-cliente su io-messaggi, evitando le problematiche relative ad attribuzioni, proiezioni  e pregiudizi. Il feedback fenomenologico si articola in 5 passi:

\begin{quote}
\textsc{Quando} io vedo che tu fai/dici, \textsc{io sento }... e allora \textsc{penso}... . Siccome \textsc{ho bisogno}... \textsc{ti chiedo}...
\end{quote}
\begin{enumerate}
\item \emph{fenomeno}: qualcosa che il counselor \textbf{osserva} nel comportamento del cliente, e su cui vuole commentare o su cui vuole che il cliente approfondisca o elabori
\item \emph{sensazione}: ciò che il counselor prova di fronte al fenomeno osservato
\item pensiero: il significato che il counselor attibuisce alla propria sensazione, la sua \index{nominalizzzioni}nominalizzazione
\item \emph{bisogno}: il bisogno che il counselor prova nel qui e ora a seguito di ciò che ha osservato, provato e pensato
\item \emph{appello}: la richiesta che il counselor fa in prima persona al cliente a seguito di quanto esposto nei punti precedenti.
\end{enumerate} 

\noindent Il feedback fenomenologico si considera già ben formato se il counselor compie i passi da 1 a 3. Una volta che il counselor ha espresso il proprio pensiero sulla sensazione prodotta dall'osservazione, è già evidente che la sensazione appartiene al counselor, non al cliente, e che eventuali richieste di chiarimento, conferma o approfondimento, quindi, non potranno essere attribuite a un suo atteggiamento pregiudiziale o giudicante. Il cliente è libero di sentire ciò che sente e di esprimerlo nel modo in cui sceglie di esprimerlo; i restanti passi possono servire a modulare l'espressione del cliente, rendendola più adeguata all'intervento di consulenza. 