%citazioni che devono comparire ma mancano ancora nel testo...
\nocite{Nanetti2003}
\nocite{Tufte2001}
\nocite{TecnicheCostruttiviste}
\nocite{fisicaefil}
\nocite{hagakure}
\nocite{blackswan}
\nocite{attili2004}
\nocite{giochi}
\nocite{carkhuff}
\nocite{gestalt}
\nocite{goleman}
\nocite{costellazioni}
\nocite{voice}
\nocite{Nanetti2007}
\nocite{May1991}
\nocite{CounselingNarrativo}
\nocite{morineau2003}
\nocite{Nanetti2006}
\nocite{Nanetti2006-2}
\nocite{cambiamento}
\nocite{momenti}
\nocite{assertivita}
\nocite{gruppo}
\nocite{potere}
\nocite{intepersonale}
\nocite{stili}
\nocite{watzlawick}




Questo piccolo lavoro assolve almeno a tre compiti, tutti egualmente parte del percorso evolutivo dell'autore:
\begin{enumerate}
\item  l'assolvimento degli obblighi per il completamento del corso triennale in Counseling Integrato a orientamento transpersonale
\item una presa di posizione contro lo \emph{Zeitgeist}, lo spirito dei tempi che in questo Paese più che in altri assume fattezze da Basso Impero
\item la scelta della propria posizione professionale fra coloro che operano delle discipline di ambito psicologico.
\end{enumerate}

\section{Una nota personale}

In un periodo storico come l'attuale, che si autodefinisce ''società della comunicazione'' ma nel quale proprio la comunicazione sembra avere perso qualsiasi rapporto fra messaggio e contenuto, fra valore intrinseco e valore di scambio, non possiamo non dirci totalmente, anzi \emph{ferocemente} d'accordo con Hillman:
\begin{quote}
Il brutto ci rende nevrotici{\ldots} In un'epoca ecologica, coraggio civile non significa soltanto esigere giustizia sociale, ma anche \emph{giustizia estetica}, e la volontà di esprimere giudizi basati sul gusto, di mettersi dalla parte della bellezza nella pubblica arena, e di parlare di tutto questo\cite{100anni}[p.156].
\end{quote}
Perciò, la scelta per questo lavoro di una forma che fosse la migliore possibile, tersa e rispettosa dell'attenzione di chi legge, va di pari passo con la mia personale scelta che anche il contenuto e l'attitudine professionale verso i metodi adottati rispondano agli stessi criteri.
Da un lato, se una forma rigorosa non garantisce la qualità dei contenuti, una forma scadente favorisce sempre un approccio grossolano e superficiale dal quale preferisco astenermi. \marginnote{Per comprendere i termini della questione, si possono mettere a confronto le esperienze di Milton Erikson e di Alejandro Jodorowsky: le tecniche adottate sono quasi identiche e la sola differenza fra i due è la diversa ispirazione, psicanalitica per il primo, esoterica per il secondo.}
Dall'altro, se molte tecniche del counseling sono comuni anche a molte \index{discipline mantiche} discipline mantiche e in generale a tutti gli usi strumentali del linguaggio quali la persuasione e la \index{propaganda} propaganda (si pensi solo al successo  della \textsc{pnl} \index{PNL} nelle tecniche di vendita e nel marketing commerciale e politico), ritengo che la dignità del counseling possa essere tutelata solo adottando una posizione rigorosamente empirista e scettica \index{empirismo!scettico} : le tecniche funzionano, qualsiasi loro razionalizzazione è gratuita o inutile.

Per la forma visuale, la scelta è caduta, giocoforza, sul sistema di composizione {\TeX}\cite{texbook}\cite{Mittelbach2004}  e su un  particolare insieme di stili ispirati agli insegnamenti del più autorevole esperto al mondo nel campo del design documentale contemporaneo, Edward R. Tufte, lo stile tipografico e visuale dei cui libri viene (per quanto possibile) riprodotto in questo lavoro\cite{Tufte1990}.

\section{Una nota metodologica}

Il rigore stilistico che mi sono dato è anche espressione di quello che metto nella mia attività professionale e ora anche nella disciplina del counseling: un approccio empiristico, di ispirazione scettica, che in particolare per quanto riguarda il counseling trovo essere il solo in grado di garantire da fughe di tipo esoterico che non intendo avallare.

Desidero operare come counselor tenendomi bene alla larga da due tendenze che considero professionalmente pericolose anche se premianti: l'arroganza epistemica e la deriva esoterico-trascendentalista.

\newthought{L'arroganza epistemica}, %
\marginnote{Il tema dell'arroganza epistemica in economia è particolarmente attuale. Secondo Taleb (citato più avanti) proprio questa arroganza (che rende i suoi portatori incapaci di ammettere limiti ai propri mezzi epistemici, ossia perfino di concepire che esistano cose che non sono in grado di conoscere) è alla base del tracollo dei mercati finanziari del 2008. Secondo Taleb, lui stesso un trader, uno dei problemi fondamentali dell'economia è l'eccesso di teoria (anzi il vero e proprio innamoramento per la teoria,  anche quando questa non sia in grado di spiegare, o addirittura ignori, fatti quotidianamente riscontrabili), a fronte di una estrema debolezza dell'impianto empirico \index{empirismo}, di difficile quando non impossibile attuazione.}%
comune fra le discipline parascientifiche, può essere descritta dall'abbandono del livello sperimentale a favore del livello linguistico: poiché infatti in tali discipline il riscontro sperimentale non può essere conclusivo né generalizzabile, il successo di una teoria si fonda sulla sua capacità di descrivere gli eventi in maniera coerente e convincente. In altre parole, si sostituisce la verifica sperimentale con la \emph{plausibilità di una narrativa}.
Si assiste, per esempio, a questo genere di comportamento in economia
dove i concetti di ''scelta razionale'' e di ''equilibrio del mercato'' sono tuttora imperanti, nonostante da decenni esistano evidenze incontrovertibili della loro completa infondatezza.

Personalmente trovo che la plausibilità narrativa \index{plausibilità narrativa} come forma di validazione sia accettabile solo nell'ambito di una pratica strettamente empiristica: adottare le pratiche che funzionano astenendosi da ogni giudizio di valore sulla teoria che le propone.
In questo consisterebbe poi la pratica del counseling a modello integrato (ma, ad esempio, anche della \textsc{pnl}): il cogliere ''fior da fiore'' le tecniche operative più utili o adatte senza curarsi di abbracciare la particolare teoria che le propone. Purtroppo, la modestia e l'onestà intellettuale necessarie per attenersi al \emph{come} ammettendo la propria ignoranza del \emph{perché} certe pratiche non sono qualità universalmente disponibili.

\newthought{La deriva esoterico-trascendentalista}, invece, sfrutta l'intrinseca limitatezza e non verificabilità delle teorie psicologiche per contrabbandare ogni sorta di concetti trascendentali o pseudo scientifici, di nuovo utilizzando la plausibilità narrativa e l'aneddotica come strumento di ''prova''. Spesso in questo campo si abusa, utilizzandoli a mo' di metafora, di concetti scientifici che per la loro complessità godono di un certo fascino popolare, nel tentativo di mutuare il rigore intellettuale ed epistemologico della loro disciplina di origine.
Concetti altrimenti rigorosi quali l'olografia, i quanti, la relatività vengono estrapolati da ogni contesto e usati in modo gratuito e metaforico mescolati in un pastone concettuale insieme a cristalli, piramidi, memoria dell'acqua, omeopatia, e fesserie \emph{New Age} assortite.

Il nostro livello di conoscenza non è ad oggi sufficiente a darci un modello della mente su basi scientifiche. Trovo quindi del tutto ingiustificato (ed intellettualmente disonesto) l'abbandono di ciò che sappiamo funzionare (il metodo scientifico) a favore di un teorizzare indiscriminato che sostituisce alle evidenze sperimentali la plausibilità narrativa e la fallacia della conferma, per quanto desiderabile e consolatorio possa essere il tipo di ''conoscenza'' che se ne può ricavare.

\newthought{In terzo luogo}, mi approccio alla disciplina del counseling alla ricerca di \emph{un} senso, non alla ricerca \emph{di Senso}. Intendo con questo che la mia visione degli uomini è quella di
\begin{quote}
macchine per la produzione di significato\cite{consciousness}[p.156]
\end{quote}
metafora potentissima quanto rigorosamente fondata sui fatti, in grado di spiegare la totalità delle ''stranezze'' della mente, dalle fallacie inferenziali\cite{hb}, agli effetti placebo, all'efficacia diffusa di discipline che vanno dall'astrologia ai tarocchi alla stessa psicologia. Alla luce di quanto sappiamo, è necessario transitare dalla limitata visione degli esseri umani come esclusivamente razionali, a una visione più problematica di esseri che, pur capaci di grande razionalità, sono più spesso per dirla con Ariely, \emph{prevedibilmente irrazionali.}\cite{predirr}

\section{La creazione del significato come atto soggettivo}

Non sarà inutile, in questa sede, ribadire l'importanza della creazione soggettiva del significato citando lo storico esempio di \textsc{eliza}\cite{eliza}, \index{\textsc{ELIZA}} il programma scritto da Joseph Weizenbaum nel 1966, che è particolarmente attinente. Tale programma, in effetti poco più che una routine di \emph{pattern-matching}, era in grado di sostenere un colloquio con un agente umano rispondendo grossomodo come un analista di ispirazione rogersiana; ecco come lo stesso Weizanbaum spiega il motivo di questa scelta\cite{eliza} (enfasi mie):

\begin{quote}
Questa modalità di conversazione è stata scelta perché il colloquio psichiatrico è uno dei pochi esempi di conversazione diadica categorizzata in linguaggio naturale dove uno dei partecipanti è libero di fingere di non sapere quasi nulla del mondo reale. Se, per esempio, un paziente dice allo psichiatra ''ho fatto un lungo giro in barca'' e quello risponde ''mi parli delle barche'', il paziente non suppone che lo psichiatra non sappia nulla di barche, ma ipotizza che abbia un qualche scopo per dirigere in questo modo la conversazione. \`{E} importante notare che questa ipotesi è fatta \emph{dal paziente}; che si tratti di un'ipotesi realistica o meno è tutt'altra questione. Ad ogni modo, [questa ipotesi] ha una utilità psicologica cruciale, poiché \emph{serve a mantenere nel paziente la convinzione} di essere ascoltato e compreso. Il paziente rinforza poi questa impressione (che perfino nel mondo reale potrebbe essere illusoria) attribuendo alla controparte ogni sorta di conoscenze pregresse, intuizioni e capacità di ragionamento. Ma, ancora una volta, questo è il contributo del paziente alla conversazione.
\end{quote}

\noindent Vediamo un esempio di ''colloquio'' fra \textsc{eliza} e un essere umano per vedere come il ''contributo del paziente'' sia talmente determinante da consentire una conversazione dotata di senso con un programma di \emph{pattern-matching} privo di qualsivoglia conoscenza del mondo\cite{eliza2}.

\begin{verse}
giovane donna: Gli uomini sono tutti uguali\\
\textsc{eliza}: In che senso?\\
giovane donna: Stanno sempre a seccarti con una cosa o l'altra.\\
\textsc{eliza}: Puoi fare un esempio specifico?\\
giovane donna: Beh, il mio ragazzo mi ha fatto venire qui.\\
\textsc{eliza}: Il tuo ragazzo ti ha fatta venire qui?\\
giovane donna: Dice che sono depressa gran parte del tempo.\\
\textsc{eliza}: Mi dispiace sentire che sei depressa.\\
giovane donna: \`{E} vero, sono infelice\\
\textsc{eliza}: Pensi che venire qui ti aiuterà a non essere infelice?\\
giovane donna: Ho bisogno di un po' di aiuto, questo è certo.\\
\textsc{eliza}: Cosa significherebbe per te trovare un po' di aiuto?\\
giovane donna: Forse potrei parlare ad andare d'accordo con mia madre.\\
\textsc{eliza}: Dimmi qualcosa della tua famiglia.\\
giovane donna: Mia madre si prende cura di me.\\
\textsc{eliza}: Chi altri nella tua famiglia si prende cura di te?\\
giovane donna: Mio padre.\\
\textsc{eliza}: Tuo padre.\\
giovane donna: Tu sei come mio padre, in un certo senso.\\
\textsc{eliza}: Che somiglianze vedi?\\
giovane donna: Non sei molto aggressivo, ma penso che non vuoi che io me ne accorga.\\
\textsc{eliza}: Cosa ti fa pensare che io non sia molto aggressivo?\\
giovane donna: Non discuti con me.\\
\textsc{eliza}: Perché pensi che io non discuta con te?
giovane donna: Hai paura di me.\\
\textsc{eliza}: Ti fa piacere pensare che io abbia paura di te?\\
giovane donna: Mio padre ha paura di tutti.\\
\textsc{eliza}: Cos'altro ti viene in mente quando pensi a tuo padre?\\
giovane donna: I bulli.\\
\textsc{eliza}: Questo cosa ha a che fare con il fatto che il tuo ragazzo ti ha fatta venire qui?\\
{\ldots}
\end{verse}

\noindent \textsc{Eliza} si rivelò sorprendentemente efficace nel sollecitare risposte emotive da parte degli utenti, che nel corso dell'interazione arrivavano ad attribuire comprensione e intenzionalità al programma, con grande sorpresa dello stesso autore che ebbe a dire\cite{weizenbaum}:

\begin{quote}
Non mi ero reso conto\ldots che anche brevi esposizioni a un programma relativamente semplice potevano indurre deliri potenti in persone altrimenti del tutto normali.
\end{quote}

\noindent Questo effetto è poi passato alla storia come \emph{effetto ELIZA} \index{\textsc{ELIZA}!effetto} quando anche altri ricercatori ebbero modi di osservare che gli utenti attribuivano inconsciamente interesse e \index{coinvolgimento}coinvolgimento emotivo alle domande di \textsc{eliza}, \emph{pur sapendo} che il programma non ne era capace\cite{billings}:

\begin{quote}
Weizenbaum fu molto turbato da ciò che scoprì con i suoi esperimenti con \textsc{eliza}: alcuni studenti mostravano un forte legame emotivo con il programma; alcuni volevano perfino essere lasciati soli [durante l'uso del programma]. Weizenbaum aveva inaspettatamente scoperto che, sebbene del tutto consci di stare parlando con un semplice programma per computer, ciononostante le persone lo trattavano come se fosse un essere reale, pensante, a cui importava dei loro problemi ---un effetto oggi noto come ''effetto \textsc{eliza}''.
\end{quote}

\newthought{Anche chi scrive} ha avuto modo di sperimentare personalmente la potenza della creazione soggettiva del significato, nel corso del laboratorio\sidenote{%
Il laboratorio è stato sviluppato per il corso residenziale di agosto 2010 per il terzo anno del corso in Counseling Integrato ad orientamento transpersonale.}%
 \emph{La risposta è dentro di te, epperò è sbagliata}, sviluppato con Simona Ortolani.
I partecipanti al laboratorio, dopo avere ricordato una situazione di conflitto ed attuato un piccolo \smallcaps{cus}, pescavano a caso un Arcano Maggiore da un insieme di 44 carte disposte in modo casuale (ma geometrico) su un tavolo. Nella enunciazione delle regole di setting, una semplice istruzione ipnotica%
\sidenote{Più propriamente, si è trattato di un \emph{ricalco-guida verbale}} aveva loro suggerito che la carta pescata \emph{sarebbe stata relativa} al conflitto da loro riportato. Come si è poi verificato, ciascuno degli oltre 20 partecipanti ha confermato che la carta \emph{effettivamente} suggeriva qualcosa di attinente alla situazione da lui precedentemente riportata.
L'insegnamento pratico tratto da questa esperienza è stato che \emph{il potere della mente umana di cogliere significati è più forte persino della loro assenza}.

Il mio status epistemologico come counselor, dunque, sarà quello di \emph{ciarlatano consapevole}, \marginnote{Il termine \emph{ciarlatano} va qui inteso non in senso spregiativo ma nel senso di persona acutamente consapevole dei limiti epistemici del proprio operare. Ad esempio si veda il grande James Randi, noto tanto quanto empirista scettico \index{empirismo!scettico} quanto come illusionista, che come professione sul proprio biglietto da visita riporta, appunto, \emph{charlatan}. Si veda al riguardo: \url{http://it.wikipedia.org/wiki/James_Randi}} ovvero di qualcuno che opera \emph{maieuticamente} con placebo, nella consapevolezza che non è il placebo in sé ad assicurare risultati, ma la convinzione del cliente di poterli raggiungere e l'abilità del terapeuta nel costruire un \emph{rapporto di empatia} \index{empatia} e fiducia nel quale il cliente possa trovare in sé energie e risorse per sviluppare questa convinzione.

\newthought{Ho scelto} perciò di improntare la mia attività a un empirismo scettico, e di mantenere sempre il profilo dell'intervento il più basso possibile, cercando di risolvere i problemi contingenti e di contribuire al cambiamento\index{cambiamento} del paziente con piccoli contributi tangibili. Questo per due motivi:

\begin{itemize}
\item l'ambito che riconosco utile e valido per l'attività del counseling non sono le risposte esistenziali, ma la ricerca di risposte pragmatiche a problemi concreti
\item le limitazioni epistemologiche, al difuori dei limitati ambiti di applicabilità del metodo scientifico, sono tali da consigliare la più grande umiltà epistemica; è in ossequio a questa che la mia attività di \emph{counselor} viene improntata alla modesta e scettica applicazione di tecniche empiricamente verificate in contesti limitati, e non alla elaborazione o alla proposta di teorie\cite{fooled}.
\end{itemize}

\noindent Sono convinto che questo atteggiamento non sia altro che la riproposizione lo spirito più profondo del counseling a modello integrato: il rifiuto di aderire a modelli teorici prefissati, in favore di una scelta personale di \textit{tecniche} della più varia ispirazione avendo come guida la sensibilità personale del counselor e il portato specifico di ciascun cliente. A ben vedere, l'adozione del ''modello integrato'' è \textit{autoreferenziale}, essendo un approccio tratto direttamente dai fondamenti operativi della \textsc{pnl}\cite{sleight} \cite{magic}.
